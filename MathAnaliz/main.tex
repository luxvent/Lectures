\documentclass[12pt, a4paper]{article}

\usepackage{graphicx}
\usepackage{amsmath, amssymb}
\usepackage[T2A]{fontenc}
\usepackage[english, russian]{babel}
\usepackage[utf8]{inputenc}
\usepackage[margin=2cm, a4paper]{geometry}
\usepackage[most]{tcolorbox}
\usepackage{caption}
\usepackage{enumitem}

\tcbuselibrary{breakable}
\tcbset{
  width=0.9\textwidth,
  halign=justify,
  center,
  breakable,
  colback=white
}

\title{Конспект по математическому анализу}
\author{Голубов Владислав}
\date{Сентябрь 2025}

\newcommand{\N}{\mathbb{N}}
\newcommand{\Q}{\mathbb{Q}}
\newcommand{\Z}{\mathbb{Z}}
\newcommand{\R}{\mathbb{R}}
\newcommand{\eps}{\varepsilon}

\newcommand{\sign}{\text{sign\ }}

\begin{document}
\maketitle
\tableofcontents
\newpage

\section{Комплексные числа}
\subsection{Определение и свойства}
\begin{tcolorbox}
\textbf{Определение.} \textit{Комплексными числами} называются числа вида $z = x + iy$, где $x, y \in \R$, а $i$ — \textit{мнимая единица}, обладающая свойством $i^2 = -1$.
\end{tcolorbox}
\begin{itemize}
    \item $x = \operatorname{Re } z$ — \textbf{действительная часть} числа $z$.
    \item $y = \operatorname{Im } z$ — \textbf{мнимая часть} числа $z$.
    \item Если $y = 0$, то $z = x$ — действительное число.
    \item Число $\overline{z} = x - iy$ называется \textbf{комплексно-сопряжённым} к $z$.
\end{itemize}

\begin{tcolorbox}
\textbf{Свойство:} $z \cdot \overline{z} = (x + iy)(x - iy) = x^2 - (iy)^2 = x^2 - i^2y^2 = x^2 + y^2$.
\end{tcolorbox}

\begin{tcolorbox}[title=Важное примечание]
\textbf{Нельзя} сравнивать комплексные числа операциями $<, >, \leq, \geq$!
\end{tcolorbox}

\subsection{Арифметические операции}
Пусть $z_1 = x_1 + iy_1$, $z_2 = x_2 + iy_2$.
\begin{enumerate}
    \item \textbf{Сложение/Вычитание:}
    $z_1 \pm z_2 = (x_1 \pm x_2) + i(y_1 \pm y_2)$
    \item \textbf{Умножение:}
    \begin{align*}
    z_1 \cdot z_2 &= (x_1 + iy_1)(x_2 + iy_2) = (x_1x_2 - y_1y_2) + i(x_1y_2 + x_2y_1)
    \end{align*}
    \item \textbf{Деление:}
    \begin{align*}
    \frac{z_1}{z_2} &= \frac{(x_1 + iy_1)(x_2 - iy_2)}{(x_2 + iy_2)(x_2 - iy_2)} = \frac{(x_1x_2 + y_1y_2) + i(x_2y_1 - x_1y_2)}{x_2^2 + y_2^2}\\
    \end{align*}
\end{enumerate}

\subsection{Формула Эйлера}
\begin{tcolorbox}
\[ e^{i\phi} = cos\phi +isin\phi \]
\[ e^z = e^{x+iy} = e^x\cdot e^{iy} = e^x \cdot (cos\; y +i\cdot sin\; y) \]
\end{tcolorbox}

\newpage
\subsection{Геометрическое представление}
\begin{figure}[h]
    \centering
    \includegraphics[width=0.4\linewidth]{images/compl_plane.png}
    \caption{Геометрическая интерпретация комплексного числа.}
    \label{fig:compl_plane}
\end{figure}
\subsection{Тригонометрическая форма}
\begin{tcolorbox}
\[ z = r(cos\phi +i\; sin\phi),\; r=|z| \]
\end{tcolorbox}
\[ z_1\cdot z_2 = r_1\cdot r_2 \cdot (cos(\phi_1+\phi_2) + i\; sin(\phi_1+\phi_2)) \]
\[\cfrac{z_1}{z_2} = \cfrac{r_1}{r_2}(cos(\phi_1-\phi_2) + i\; sin(\phi_1-\phi_2)) \]

\subsection{Формула Де-Муавра}
\begin{tcolorbox}
\[ (cos \phi +i\; sin\phi)^k = cos\; k\phi + i\; sin\; k\phi \]
\end{tcolorbox}

\subsection{Комплексные корни}
\[ \sqrt[n]{z} = \omega \]
\[ \omega^n = z, z \not= 0\]
\[ z = re^{i\phi}, \omega = \rho e^{i\Psi}  \]
\[ \omega^n = \rho^n e^{in\Psi} = z = re^{i\phi} = re^{i(\phi+2\pi k)}\]
\begin{tcolorbox}
\[ \rho^n = r \Rightarrow \rho = \sqrt[n]{r} \]
\[ n\Psi = \phi + 2\pi k \Rightarrow \Psi = \cfrac{\phi}{n} + \cfrac{2\pi}{n}k \]
\end{tcolorbox}
Корни будут образовывать правильный многоугольник.

\section{Метод математической индукции (ММИ)}
\begin{tcolorbox}
\textbf{Алгоритм доказательства по индукции:}
\begin{enumerate}
    \item \textbf{База индукции:} Проверить утверждение для $n = 1$.
    \item \textbf{Индукционное предположение:} Предположить, что утверждение верно для $n = k$.
    \item \textbf{Индукционный переход:} Доказать, что из этого следует верность утверждения для $n = k+1$.
\end{enumerate}
\end{tcolorbox}

\subsection{Бином Ньютона}
\begin{tcolorbox}
\textbf{Определение.} \textit{Биномиальный коэффициент}:
$C_n^k = \dfrac{n!}{k!(n-k)!}$, где $n,k \in \mathbb{N}_0$, $0! = 1$.
\end{tcolorbox}

\begin{tcolorbox}
\textbf{Формула бинома Ньютона:}
$$(a+b)^n = \sum_{k=0}^{n} C_n^k a^{n-k}b^k$$
\end{tcolorbox}

\begin{tcolorbox}[title=Доказательство по ММИ, breakable]
\textbf{База индукции:} Для $n=1$:
$$(a+b)^1 = a + b$$
$$\sum_{k=0}^{1} C_1^k a^{1-k}b^k = C_1^0 a^1 b^0 + C_1^1 a^0 b^1 = 1 \cdot a \cdot 1 + 1 \cdot 1 \cdot b = a + b$$
База индукции доказана.

\textbf{Индукционное предположение:} Предположим, формула верна для $n = m$:
$$(a+b)^m = \sum_{k=0}^{m} C_m^k a^{m-k}b^k$$

\textbf{Индукционный переход:} Докажем для $n = m+1$.
Умножим обе части предположения на $(a+b)$:
$$(a+b)^{m+1} = (a+b) \cdot \sum_{k=0}^{m} C_m^k a^{m-k}b^k$$
Раскроем скобки:
$$= \sum_{k=0}^{m} C_m^k a^{m+1-k}b^k + \sum_{k=0}^{m} C_m^k a^{m-k}b^{k+1}$$
Во второй сумме сделаем замену индекса $j = k+1$:
$$= \sum_{k=0}^{m} C_m^k a^{m+1-k}b^k + \sum_{j=1}^{m+1} C_m^{j-1} a^{m+1-j}b^{j}$$
Теперь объединим суммы, выделяя крайние слагаемые:
$$= C_m^0 a^{m+1} + \sum_{k=1}^{m} \left[ C_m^k + C_m^{k-1} \right] a^{(m+1)-k}b^k + C_m^m b^{m+1}$$
Используем свойство биномиальных коэффициентов:
$$C_m^k + C_m^{k-1} = C_{m+1}^k$$
Учитывая, что $C_m^0 = C_{m+1}^0 = 1$ и $C_m^m = C_{m+1}^{m+1} = 1$, получаем:
$$(a+b)^{m+1} = \sum_{k=0}^{m+1} C_{m+1}^k a^{(m+1)-k}b^k$$
Индукционный переход завершён.
\end{tcolorbox}

\subsection{Неравенство о средних}
\begin{tcolorbox}
\textbf{Теорема (Неравенство между средним арифметическим и средним геометрическим):}
Для любых $a_1, a_2, \dots, a_n \geq 0$ справедливо:
$$\frac{a_1 + a_2 + \dots + a_n}{n} \geq \sqrt[n]{a_1 a_2 \dots a_n}$$
Равенство достигается тогда и только тогда, когда $a_1 = a_2 = \dots = a_n$.
\end{tcolorbox}

\begin{tcolorbox}[title=Доказательство по ММИ (метод Коши / метод обратой индукции), breakable]
Докажем теорему в три этапа.

\textbf{1. База индукции для степеней двойки ($n=2^m$).}
\begin{itemize}
    \item \textbf{Для $n=2$:} Докажем $\frac{a_1 + a_2}{2} \geq \sqrt{a_1 a_2}$.
    $$(a_1 - a_2)^2 \geq 0 \Rightarrow a_1^2 - 2a_1a_2 + a_2^2 \geq 0 \Rightarrow a_1^2 + 2a_1a_2 + a_2^2 \geq 4a_1a_2 \Rightarrow$$
    $$\Rightarrow (a_1 + a_2)^2 \geq 4a_1a_2 \Rightarrow \frac{a_1 + a_2}{2} \geq \sqrt{a_1 a_2}$$
    \item \textbf{Предположим, неравенство верно для $n = k$.}
    \item \textbf{Докажем для $n = 2k$:}
    $$\frac{a_1 + \dots + a_{2k}}{2k} = \frac{\frac{a_1 + \dots + a_k}{k} + \frac{a_{k+1} + \dots + a_{2k}}{k}}{2} \geq$$
    $$\geq \frac{\sqrt[k]{a_1 \dots a_k} + \sqrt[k]{a_{k+1} \dots a_{2k}}}{2} \geq \sqrt{\sqrt[k]{a_1 \dots a_k} \cdot \sqrt[k]{a_{k+1} \dots a_{2k}}} = \sqrt[2k]{a_1 \dots a_{2k}}$$
\end{itemize}

\textbf{2. Докажем, что если неравенство верно для $n$, то оно верно и для $n-1$.}
Рассмотрим $a_1, a_2, \dots, a_{n-1} \geq 0$. Пусть
$$a_n = \frac{a_1 + a_2 + \dots + a_{n-1}}{n-1}$$
Для набора из $n$ чисел неравенство верно:
$$\frac{a_1 + \dots + a_{n-1} + a_n}{n} \geq \sqrt[n]{a_1 a_2 \dots a_{n-1} a_n}$$
Подставим $a_n$:
$$\frac{(a_1 + \dots + a_{n-1}) + \frac{a_1 + \dots + a_{n-1}}{n-1}}{n} = \frac{a_1 + \dots + a_{n-1}}{n-1} = a_n$$
Таким образом:
$$a_n \geq \sqrt[n]{a_1 a_2 \dots a_{n-1} a_n}$$
Возведём в степень $n$:
$$a_n^n \geq a_1 a_2 \dots a_{n-1} a_n \Rightarrow a_n^{n-1} \geq a_1 a_2 \dots a_{n-1}$$
Извлекая корень $(n-1)$-й степени:
$$a_n \geq \sqrt[n-1]{a_1 a_2 \dots a_{n-1}} \Rightarrow \frac{a_1 + \dots + a_{n-1}}{n-1} \geq \sqrt[n-1]{a_1 a_2 \dots a_{n-1}}$$

\textbf{3. Завершение доказательства.}
Мы доказали, что:
\begin{enumerate}
    \item Неравенство верно для $n=2$ (а значит, для $n=4,8,16,\dots$)
    \item Из верности для $n$ следует верность для $n-1$
\end{enumerate}
Следовательно, неравенство верно для любого натурального $n$.
\end{tcolorbox}

\subsection{Неравенство треугольника}
\begin{figure}[h]
    \centering
    \includegraphics[width=0.5\linewidth]{images/Неравенство(вектора).png}
    \caption{Геометрический смысл неравенства треугольника: длина стороны \( |\vec{a} + \vec{b}| \) не превосходит суммы длин сторон \( |\vec{a}| + |\vec{b}| \).}
    \label{fig:triangle}
\end{figure}
\begin{tcolorbox}
\textbf{Теорема (Неравенство треугольника):}
Для любых комплексных чисел $z_1, z_2$ справедливо:
$$|z_1 + z_2| \leq |z_1| + |z_2|$$
\end{tcolorbox}

\begin{tcolorbox}[title=Доказательство, breakable]
$|a+b| \leq |a| + |b|$
\begin{enumerate}
    \item $a \geq 0\ (|a| \geq |b|)$ \\
    $a + b \geq 0$, то $|a+b| = a+b$.\\
    $a + b \leq 0$, то $|a+b| = -(a+b) = -a-b \leq |a|+|b|$
    \item $|a-b| \geq ||a|-|b||$\\
    $a = (a - b) + b$ по н.т.: $|a + 0| \leq |a-b|+|b| \Rightarrow |a-b| \geq |a|-|b|$ \\
    Аналогично $|b| \leq |b-a|+|a| \Rightarrow |a-b| \geq |b|-|a|$ \\
    Получим, что 
    $
    \begin{cases}
        |a-b| \geq |a|-|b|\\
        |a-b| \geq |-(|a|-|b|)
    \end{cases}
    \Rightarrow |a-b| \geq ||a|-|b||
    $
\end{enumerate}
\end{tcolorbox}

\begin{tcolorbox}[title=Следствие, breakable]
$$||z_1| - |z_2|| \leq |z_1 \pm z_2| \leq |z_1| + |z_2|$$
\end{tcolorbox}

\section{\S I. Пределы и числовая система}
\subsection{Пределы последовательности}
\begin{tcolorbox}
\textbf{Определение.} Число $a$ называется \textit{пределом последовательности} $\{a_n\}$, если
$$\forall \eps > 0\ \exists N_\eps \in \N\ \forall n > N_\eps: |a_n - a| < \eps$$
Обозначение: $\lim_{n\to\infty} a_n = a$ или $a_n \to a$ при $n \to \infty$.
\end{tcolorbox}

\begin{tcolorbox}[title=Теорема о единственности предела, breakable]

\textbf{Теорема:} Последовательность не может иметь более одного предела.

\textbf{Доказательство от противного:}
Предположим, что последовательность $\{a_n\}$ имеет два различных предела: $a_n \to A$ и $a_n \to B$, где $A \neq B$.
Пусть $\eps = \frac{|A - B|}{4} > 0$. Тогда по определению предела:
\begin{itemize}
    \item $\exists N_1: \forall n > N_1: |a_n - A| < \eps$
    \item $\exists N_2: \forall n > N_2: |a_n - B| < \eps$
\end{itemize}
Возьмём $n > \max(N_1, N_2)$. Тогда выполняются оба неравенства. Оценим разность $|A - B|$:
$$|A - B| = |(A - a_n) + (a_n - B)| \leq |A - a_n| + |a_n - B| < \eps + \eps = 2\eps = \frac{|A - B|}{2}$$
Получили противоречие: $|A - B| < \frac{|A - B|}{2}$. Следовательно, наше предположение неверно, и предел единственен.
\end{tcolorbox}

\subsection{Что-то про доказательство существования иррациональных чисел (про $\sqrt{2}$).1}

$\sqrt{2} = \frac{p}{q}$, $p, q \in N$\\
$\cfrac{p^2}{q^2} = 2 \Rightarrow p^2 = 2q^2$ Пусть: $p = 2k \Rightarrow 4k^2 = 2q^2 \Rightarrow q^2 = 2k^2$

\newpage
\subsection{Ограниченные и сходящиеся последовательности}

\begin{tcolorbox}
\textbf{Определение}
$\{a_n\}$  \textit{ограничена}, если $\exists M>0: \forall n\ |a_n|<M,\ a_n, M, n \in Q$
\end{tcolorbox}

\begin{tcolorbox}
\textbf{Определение}
$\{a_n\}$  \textit{не ограничена}, если $\forall M>0: \exists n\ |a_n|\geq M,\ a_n, M, n \in Q$ 
\end{tcolorbox}

\begin{tcolorbox}
\textbf{Теорема 2} \textit{Любая сходящаяся последовательность ограничена} \\
Если $\{a_n\}$ сходиться $\Rightarrow \{a_n\}$ ограничена
\end{tcolorbox}

\begin{tcolorbox}[title=Доказательство Т2, breakable]
\[\eps := 1\ \exists N: \forall n > N\ \ |a_n-a| < 1 \Leftrightarrow -1<a_n-a<1 \Leftrightarrow a-1 < a_n < a+1  \]
\[ M:= max\{ |a_1|, |a_2|, ..., |a_N|, |a-1|, |a+1| \} + 1\]
$\Rightarrow \{a_n\}$ - ограничена (сверху)
\end{tcolorbox}

\begin{tcolorbox}
\textbf{Определение}
$\{a_n\}$  \textit{ограничена сверху, если} $\exists M: \forall n$ $a_n < M$
\end{tcolorbox}

\begin{tcolorbox}
\textbf{Определение}
$\{a_n\}$  \textit{ограничена снизу, если} $\exists m: \forall n$ $a_n < m$
\end{tcolorbox}

Пример:
\begin{enumerate}
    \item $\{\cos n\}\ |\cos n| \leq 1$
    \item $\{n\}$ ограничена снизу но не сверху (0, 1, ...)
\end{enumerate}

\newpage
\subsection{Бесконечно малые последовательности}

\begin{tcolorbox}
\textbf{Определение}
\textit{Бесконечно малые последовательности}
\begin{center}
    $\{\alpha_n\}\ (\forall n, \alpha_n \in Q)$ бесконечно малая, если $\alpha_n 
    \xrightarrow{n\rightarrow \infty} 0$    
\end{center}
\end{tcolorbox}

\begin{tcolorbox}
\textbf{Теорема 3} 
\begin{center}
    Если $a_n \rightarrow a \Leftrightarrow a_n = a + \alpha_n$, где $ \{\alpha_n\}$ - б.м.
\end{center}
\end{tcolorbox}

\begin{tcolorbox}[title=Доказательство Т3]
$\Rightarrow$:\\
\[ \forall \eps>0\ \exists n_\eps : \forall n > n_\eps\ |a_n - a| < \eps \]
\begin{center}
    Пусть $ |a_n-a| = \alpha_n \Rightarrow a_n = a + \alpha_n $    
\end{center}
\[ \forall \eps > 0\ \exists n_\eps: \forall n > n_\eps\  |a_n -a| = |\alpha_n - 0| < \epsilon \Leftrightarrow \alpha_n \xrightarrow[n\rightarrow \infty]{} 0 \]
$\Leftarrow$:\\
\begin{center}
    $ \{\alpha_n \} $ - б.м., т.е. $ \forall \eps>0\ \exists n_\eps : \forall n > n_\eps\ \eps > |\alpha_n| = |a_n-a| \Leftrightarrow a_n \xrightarrow[n \rightarrow \infty]{} a$    
\end{center}
\end{tcolorbox}

\begin{tcolorbox}
\textbf{Теорема 4}
\begin{enumerate}
    \item $\{\alpha_n\}$ и $\{\beta_n\}$ - б.м. $\Rightarrow \{\alpha_n \pm \beta_n\}$ - б.м.
    \item $\{\alpha_n\}$ - б.м. и $\{\beta_n\}$ - ограничена $\Rightarrow \{\alpha_n \cdot \beta_n\}$ - б.м.
\end{enumerate}
\end{tcolorbox}

\begin{tcolorbox}[title=Доказательство Т4: Предел суммы/разности б.м. последовательностей, breakable]
\small

\textbf{Доказательство:}
\begin{enumerate}

\item[I] Требуется доказать, что \( \forall \varepsilon > 0\ \exists n_\varepsilon \in \mathbb{N} : \forall n > n_\varepsilon\ |\alpha_n \pm \beta_n| < \varepsilon \).

\begin{enumerate}
    \item[1.] Зафиксируем произвольное \( \varepsilon > 0 \).
    \item[2.] Так как \( \{\alpha_n\} \) — б.м., то для числа \( \frac{\varepsilon}{2} > 0 \) найдётся номер \( n_{\varepsilon}' \) такой, что: $\forall n > n_{\varepsilon}'\quad |\alpha_n| < \frac{\varepsilon}{2}$
    
    \item[3.] Аналогично $\forall n > n_{\varepsilon}''\quad |\beta_n| < \frac{\varepsilon}{2}$
    
    \item[4.] Выберем номер \( n_\varepsilon = \max\{n_{\varepsilon}', n_{\varepsilon}''\} \). Тогда для всех \( n > n_\varepsilon \) будут выполняться \textbf{оба} неравенства из пунктов (2) и (3).
    \item[5.] Оценим модуль суммы (или разности) для всех \( n > n_\varepsilon \), используя неравенство треугольника:
    \[
    |\alpha_n \pm \beta_n| \leq |\alpha_n| + |\beta_n| < \frac{\varepsilon}{2} + \frac{\varepsilon}{2} = \varepsilon
    \]
\end{enumerate}

Так как \( \forall \varepsilon > 0\ \exists n_\varepsilon: \forall n > n_\eps\ |\alpha_n \pm \beta_n| < \varepsilon \), то по определению последовательность \( \{\alpha_n \pm \beta_n\} \) является бесконечно малой.

\item[II] {$\beta_n$} - ограничена $\Rightarrow \exists M > 0:\ \forall n \ |b_n| < M \ \forall\eps > 0\ \exists n_\eps : \forall n > n_\eps\ |\alpha_n| < \cfrac{\eps}{2}$\\
$|\alpha_n \cdot \beta_n|= |\alpha_n| \cdot |\beta_n| \leq \cfrac{\eps}{M} \cdot M = \eps$
\end{enumerate}
\end{tcolorbox}

\begin{tcolorbox}
\textbf{Теорема 5}
\[ a_n \xrightarrow[n\rightarrow \infty]{}a,\ b_n \xrightarrow[n\rightarrow \infty]{}b\]
\begin{enumerate}
    \item $a_n + b_n \xrightarrow[n\rightarrow \infty]{} a + b$
    \item $a_n \cdot b_n \xrightarrow[n\rightarrow \infty]{} a \cdot b$
    \item Если $b_n \ne 0\ \forall n \ nb \ne 0$, то $\cfrac{a_n}{b_n} \xrightarrow[n\rightarrow \infty]{} \cfrac{a}{b}$
\end{enumerate}
\end{tcolorbox}

\begin{tcolorbox}[title=Доказательство Т5: Арифметические свойства предела, breakable]
\textbf{Из Т4 про арифметические свойства б.м. последовательностей}
\begin{enumerate}
    \item $a_n + b_n = (a + \alpha_n) + (b+\beta_n) = (a+b)+(\alpha_n+\beta_n)$\\
    $(\alpha_n+\beta_n)$ - Сумма б.м., $(a+b)+(\alpha_n+\beta_n) = $ по Т3 $= a_n+b_n \rightarrow a + b$
    \item $a_nb_n = (a + \alpha_n) + (b + \beta_n) = ab + (\alpha_nb + \beta_na + \alpha_n\beta_n) \xrightarrow[]{T3} ab$
    \item Докажем, что $\cfrac{a_n}{b_n} - \cfrac{a}{b}$ - б.м.\\
    $\cfrac{a_n}{b_n} - \cfrac{a}{b} = \cfrac{a_nb-ab_n}{b_nb} = \cfrac{(a+\alpha_n)b-a(b+\beta_n)}{bb_n} = \cfrac{\alpha_nb-a\beta_n}{bb_n}$\\
    $\alpha_nb-a\beta_n$ - бесконечно малая\\
    Проверим ограниченность $\cfrac{1}{bb_n}$\\
    $\eps=\cfrac{|b|}{2}: \exists n_\eps\ \forall n>n_\eps\ |b_n-b|<\cfrac{\eps}{2}$\\
    $|b_n| = |b-(b-b_n)| \geq{}$(неравенство треугольника)$\geq ||b|-|b-b_n||\geq|b|-|b-b_n|>\cfrac{|b|}{2}\Rightarrow\cfrac{1}{|b_n|}<\cfrac{2}{|b|} \Rightarrow \{\cfrac{1}{b_n}\}$ - ограничена\\
    Значит, что $\cfrac{a_n}{b_n} - \cfrac{a}{b} = \cfrac{\alpha_nb-a\beta_n}{bb_n}$ - бесконечно малая.
\end{enumerate}
\end{tcolorbox}

\newpage
\section*{\S 2 Дедекиндовы сечения по видеозаписи}
\subsection*{Неполнота рациональных чисел.}
$r = \cfrac{p}{q},p\in \Z,q\in\N $\\
Дробь можно сделать несократимой.\\
Пусть $(\cfrac{p}{q})^2 = 2$ - несократимая дробь $\Rightarrow p^2 = 2q^2 \Rightarrow p = 2k$ (p - чётное число) \\
$k^2 = 2q^2 \Rightarrow q^2 = 2k^2 \Rightarrow $ q - чётное число
\\т.к. дробь несократимая, а числитель и знаменатель чётные, то она на самом деле сократимая. Противоречие! \\
Значит это число \textbf{нельзя} представить рациональной дробью.
\begin{figure}[h]
    \includegraphics[width=0.2\linewidth]{images/Дедекиндовы сечения/Начало дедекиндовых сечений.png}
\end{figure}\\
Если на оси отметить все рациональные числа точками, то $\sqrt{2}$ - будет выколотой точкой.
\subsection*{Отрезки}
$I_n = [a_n, b_n] = \{ r \in \Q\; |\; a_n \leq r \leq b_n \}$, где $a_n, b_n \in \Q$
\begin{tcolorbox}
Будем считать, что $\forall n [a_{n+1}, b_{n+1}] \subset [a_n, b_n]$ - \textbf{вложенные отрезки.}\\
Это значит, что $a_n \leq a_{n+1} \leq b_{n+1} \leq b_n$ и следующий отрезок меньше предыдущего.
\end{tcolorbox}
\begin{tcolorbox}
    $b_n - a_n \xrightarrow{n\to\infty} 0$ - \textbf{стягивающиеся отрезки.}
\end{tcolorbox}
\textbf{Дальше будем подразумеваться}, что все последовательности $\{I_n\}$ - вложенные и стягивающиеся в точку.
\begin{figure}[h]
    \includegraphics[width=0.25\linewidth]{images/Дедекиндовы сечения/Смысл отрезков.png}
\end{figure}\\
Что нам дают такие отрезки: $\forall r$ \\
\begin{equation}
    \exists n: r < a_n \Rightarrow \forall m > n:\; r < a_m
\end{equation}
\begin{equation}
    \exists n: r > b_n \Rightarrow \forall m > n:\; r > b_m
\end{equation}
\begin{equation}
    \forall n: a_n \leq r \leq b_n
\end{equation}
\begin{enumerate}
    \item левый класс для $\{I_n\}$ (всегда не пуст)
    \item правый класс для $\{I_n\}$ (всегда не пуст)
    \item центральный класс для $\{I_n\}$ (может быть пустым учитывая $r \in \Q\$)$. Не может содержать более 1 Q числа.
\end{enumerate}
Слово "класс" подразумевает множество.
\begin{tcolorbox}
    \textbf{Дедекиндово сечение}  - множество рациональных чисел (Q) порождённое последовательностью $\{I_n\}$.
\end{tcolorbox}
Тогда каждому действительному числу будет соответствовать своё дедекиндово сечение.

% \section{\S 2 Дедекиндовы сечения}
% $\sqrt{2}\in Q$\\
% $I_1 = [a_1, b_1],\ a_i,b_j\ \in Q$\\
% $I_n = [a_n, b_n]$\\
% $[a_1, b_1] \supset [a_2,b_2] \supset \dots \supset [a_n,b_n] \supset \dots$\\
% $[a_{n+1},b_{n+1}] \subset [a_n,b_n]$\\
% Длина $I_n = b_n - a_n \rightarrow 0$. Отрезки невырожденные. \\
% $b_n \rightarrow a_n \rightarrow 0$\\
% $\{ [a_n,b_n] \}$ - последовательность вложенных стягивающихся отрезков.\\
% Не может быть 2х точек, может быть только 1 и 0 точек.

% $\{ [a_n,b_n] \}\ \forall r \in Q$
% \begin{enumerate}
%     \item  $\exists n`\ r<a_{n`} \Rightarrow \forall m 
%     \geq n`\;\;\; r < a_m$
%     \item $\forall n\ a_n \leq r \leq b_n$ ?
%     \item  $\exists n``:\ r > b_{n``} \Rightarrow \forall m \geq n``\;\;\; r>b_m$
% \end{enumerate}

% \begin{tcolorbox}
% \textbf{Определение}\\
% \textit{Множество всех $r \in Q$ для которых 
% \begin{enumerate}
%     \item левый класс для {$I_n$}
%     \item Если $\exists r \in Q$ для которого 2) центр класса для {$I_n$}
%     \item Множество всех всех $r \in Q$
%     \item -правый класс для {$I_n$}
% \end{enumerate}
% }
% \end{tcolorbox}

% \begin{enumerate}
%     \item $\forall r`$ из 1., $\forall r``$ из 3.  $r`<r``$
%     \item Центральный класс может быть пуст.
% \end{enumerate}

% \begin{tcolorbox}
% \textbf{Определение}\\
% \textit{Такое разделение на классы - \textbf{Дедекиндово сечение} Q, порождённый $\{I_n\}$ }
% \end{tcolorbox}

\begin{tcolorbox}
\textbf{Определение}\\
\textit{$\{I_n\}$ и $\{I_n``\}$ - \textbf{эквиваленты}, если они порождают одинаковые разбиения на классы}
\end{tcolorbox}

\begin{tcolorbox}
\textbf{Теорема 1}\\
$\{ I_n \}\sim$ эквивалентна $\{ I_n` \} \Leftrightarrow$
\begin{enumerate}
    \item $\forall n \;\; a_n - a_n` \xrightarrow{n \to \infty} 0 $\\
    ИЛИ
    \item $\forall n a_n \leq b_n`,\; a_n` \leq b_n$
\end{enumerate}
\end{tcolorbox}

\begin{tcolorbox}[title=Доказательство Т1, breakable]
\includegraphics[width=0.4\linewidth]{images/Дедекиндовы сечения/Доказательство Т1.png}\\
% \begin{figure}[h]
%     \centering
%     \includegraphics[width=0.5\linewidth]{images/Дедекиндовы сечения/Доказательство Т1.png}
% \end{figure}\\
\textbf{Только п.1.} \\
$\Rightarrow$: $\{ I_n \} \sim \{ I_n` \} \Rightarrow a_n-a_n` \rightarrow 0$\\
от противного: тогда $a_n-a_n` \nrightarrow 0 \Leftrightarrow \\ 
\exists \eps>0:\; \forall N\;\; \exists n > N\; |a_n - a_n`| \geq \eps $ \\
$\Rightarrow $ для бесконечно многих номеров либо $a_n-a_n` > \eps$, либо $a_n` - a_n > \eps$\\
Пусть для бесконечно многих номеров $a_n - a_n` > \eps$ \\
$\exists n_\eps$ длина $[a_n`, b_n`] < \cfrac{\eps}{2}$\\
$b_n`-a_n` \rightarrow 0 \; \exists n_\eps: \; \forall n > n_\eps\; |b_n`-a_n`| < \cfrac{\eps}{2}$\\
Т.к. $\exists r \in [b_n`, a_n]$, то она принадлежит правому классу $\{I_n`\}$ и левому классу $\{I_n\}$. Последовательности не эквивалентны.\\

$\Leftarrow: a_n - a_n`\; \xrightarrow{n\rightarrow \infty} 0 \Rightarrow \{ I_n \} \sim \{ I_n` \}$\\
От противного: пусть $\{ I_n \} \nsim \{ I_n` \}$, то есть $\exists r \in Q\; r$ из левого класса для одной и центрального или правого класса другой.

r из левого класса $\{ I_n \} \exists n: r < a_n$
\begin{enumerate}
    \item[i.] r из правого класса $\{ I_n` \} \Rightarrow \exists n`: \forall n > n`\;\; a_n` \leq b_n` < r < a_n \leq a_m$\\
    \item[ii.] r из центрального класса $\{ I_n` \}$\\
    $\exists n: r < a_n$ Пусть $\eps = a_n - r(>0)$\\
    $\exists n_\eps: \forall m > n_\eps |b_m`-a_m`| < \cfrac{\eps}{2} \Rightarrow [a_m`,b_m`]$ на расстоянии не меньше $\cfrac{\eps}{2}$ от $a_n$
\end{enumerate}

\textbf{2 вариант доказательства в обратную сторону.}\\
$\Leftarrow: a_n - a_n`\; \xrightarrow{n\rightarrow \infty} 0 \Rightarrow \{ I_n \} \sim \{ I_n` \}$\\
а) совпадение левых классов $r \in $ левый класс для $\{I_n\}$
\includegraphics[width=0.4\linewidth]{images/Дедекиндовы сечения/Доказательство Т1.2.png}\\
$\exists n: r < a_n$\\
$\exists n_\eps: \forall n>n_\eps \;\;|a_n-a_n`| < \cfrac{\eps}{2} \Rightarrow r < a_n` \Rightarrow r\ \in $ левый класс для $\{I_n`\}$ \\
б) правые классы аналогично
\end{tcolorbox}
\subsubsection{Ограничение на количество элементов центрального класса.}
Если r из центрального класса, то $\forall n: \; a_n \leq r \leq b_n$\\
А если $\exists r`,r \in $ центральный класс $r<r`$, то $a_n \leq r < r` \leq b_n$. Тогда длина отрезка не может быть меньше длины отрезка $[r,r`]$ значит она не стремится к 0.\\
В центральном классе может быть либо ё число либо 0.
\subsubsection{Пример 2-х последовательностей стягивающихся к 0}
$\{I_n\} = [\cfrac{1}{2n},\cfrac{1}{2n}]\;\; \{I_n`\} = [\cfrac{1}{2n+1},\cfrac{1}{2n+1}] \;\; \{I_n``\} = [0,\cfrac{1}{2n+1}]$ Они определяют 1 и то же число.
\subsubsection{Пример 2-х последовательностей стягивающихся к $\sqrt{2}$}
$[\sqrt{2} - \cfrac{1}{n}, \sqrt{2} + \cfrac{1}{n}]$
\subsection{Действительные числа}
\begin{tcolorbox}
\textit{\textbf{Действительные числа} - это вложенные стягивающиеся отрезки с рациональными концами. Числа равны, если последовательность $\{I_n\} \sim \{I_n`\}$  }\\
\textit{\textbf{Действительные число} - отождествляется с дедекиндовым сечением, порождённым $\{[a_n, b_n]\}$ . Числа равны, если последовательность $\{[a_n, b_n]\} \sim \{[a_n`, b_n`]\}$  }
\end{tcolorbox}

\subsection{Сравнение действительных чисел и арифметические операции}
$r, p, q \in \Q$\\
$x, y, z \in \R$
\subsubsection{Определение сравнения рационального и действительного числа.}
\begin{tcolorbox}
\begin{enumerate}
    \item $r < x$, если r из левого класса $\{ [a_n,b_n] \}$
    \item $r \equiv x$, если r из центрального класса $\{ [a_n,b_n] \}$
    \item $r > x$, если r из правого класса $\{ [a_n,b_n] \}$
\end{enumerate}
% Я подумал и я решил рассказать всё по другому.
\end{tcolorbox}
Нет зависимости от конкретного выбора $\{ [a_n,b_n] \}$\\
\subsubsection{Определение сравнения 2-х действительных числе (запись лекции)}
\begin{tcolorbox}
    $x, y \in \R$
    \begin{enumerate}
    \item $x \not= y \Leftrightarrow \exists r_1 \in $ левый класс для x\\
    $ r_1 \in $ правый класс для y.\\
    Либо\\
    $\exists r_2 \in $ Правый класс для x, $r_2 \in$ Левый класс для y.\\
    \includegraphics[width=0.5\linewidth]{images/Действительные числа/Сравнение действительных чисел.png}
    \includegraphics[width=0.5\linewidth]{images/Действительные числа/Сравнение действительных чисел2.png}\\
    $c_n < d_n < d_k < r_1 < a_m \leq a_n < b_n,\; \forall n > k+m$
        \item $x < y$, если $\exists r \in \Q\\ x < r < y$
        \item $x > y$, если $\exists r \in \Q\\ x > r > y$
        \item Транзитивность $x < y,\; y < z \Rightarrow x < z:\\
        \exists r_1 \in \Q,\; \exists r_2 \in \Q: x < r_1 < y < r_2 < z$
    \end{enumerate}
\end{tcolorbox}

$\forall x, y \in \R$ если левые и правые классы совпадают $\Rightarrow x=y$ для x и y.\\
Если не совпадают, то либо $\exists r \in \Q, r \in $ левому классу для y и правому классу для x.\\
$x \leq r \leq y \Rightarrow x < y$\\
либо $\exists r \in Q: r \in $ правому классу для y и правому классу для x.\\
$x < r < x \Rightarrow y < x$

\begin{tcolorbox}
    Условия:
    \begin{enumerate}
        \item Если $r < x \Leftrightarrow r < a_n$ для достаточно больших номеров(n)
        \item Если $r = x \Leftrightarrow \forall n\; a_n \leq r \leq b_n$
        \item Если $r < x \Leftrightarrow r > b_n $ для достаточно больших номеров(n)
    \end{enumerate}    
\end{tcolorbox}


\begin{tcolorbox}[title=Доказательство 2., breakable]
\begin{enumerate}
    \item Если $r < x \Leftrightarrow r $ из левого класса для $\{ [a_n,b_n] \} \Leftrightarrow \exists n \;\; r < a_n$
    \item аналогично 
    \item не 1. и не 2.
\end{enumerate}
\end{tcolorbox}

\begin{tcolorbox}
    $x, y \in R\; x < y$, если $\exists r \in Q: x < r < y$
\end{tcolorbox}

\subsection{Арифметические операции}
$x, y\in \R\;\;\;\;\; x \sim \{[a_n,b_n]\}$\\
\textbf{Действительное число} - класс эквивалентности стягивающихся отрезков с иррациональными концами.\\
По простому у каждого такого класса есть рациональное число - центральный класс. \\
\[x + y :\sim \{[a_n+c_n,b_n+d_n]\} \]
\begin{tcolorbox}[title=Определим операцию сложения]
    \begin{enumerate}
        \item $\{[a_n+c_n,b_n+d_n]\}$ - вложенные стягивающиеся отрезки.
        $\forall n\;\; a_n\leq a_{n+1} < b_{n+1} \leq b_n$\\
        $\forall n\;\; c_n\leq c_{n+1} < d_{n+1} \leq d_n$\\
        $a_n+c_n\leq a_{n+1} + c_{n+1} < b_{n+1} + d_{n+1} \leq b_n + d_n$\\
        $[a_{n+1}+c_{n+1},b_{n+1}+d_{n+1}] \subset [a_n+c_n,b_n+d_n]$ - вложенность.\\
        $(d_n+d_n) - (a_n+c_n) = (b_n - a_n) + (d_n - c_n) \to 0$ - стягиваемость.
        \item Если $\{[a_n,b_n]\} \sim \{[a_n`,b_n`]\}$ и $\{[c_n,d_n]\} \sim \{[c_n`,d_n`]\} \Rightarrow \{[a_n+c_n,b_n+d_n]\} \sim \{[a_n`+c_n`,b_n`+d_n`]\}$\\
        $a_n - a_n` \to 0 \;\;\; c_n-c_n` \to 0$\\
        Достаточно проверить, что $(a_n + c_n) - (a_n` + c_n`) \to 0$
        \item Если $x,y \in \Q$\\
        Пусть $x \sim \{ [x-\cfrac{1}{n}, x+\cfrac{1}{n}] \}$, $y \sim \{ [y-\cfrac{1}{n}, y+\cfrac{1}{n}] \}$\\
        \[ x+y \sim \{x+y-\cfrac{2}{n};x+y+\cfrac{2}{n}\} \]
        x+y принадлежит центральному классу. Зачит корректно.
    \end{enumerate}
\end{tcolorbox}
Свойства:
\begin{enumerate}
    \item $ x + y = y + x $ - из определения.
    \item $ (x + y) + z = x + (y + z) $ - из определения.
    \item[3.] x - y :$\sim \{[a_n-c_n, b_n-d_n]\}$ - это неправильно!
    \item x - y :$\sim \{[a_n-d_n, b_n-c_n]\}$
    \item $x\cdot y,\;\;x>0,y>0$\\
    $x\cdot y \sim \{[a_nc_n;b_nd_n]\}$, а концы могут быть какими угодно. Проблемы со знаком. Тогда пусть $a_1>0,\;c_1>0$. Доказывать всё не будем. Надо понять, что оно доказывается просто перебором всех случаев.
\end{enumerate}

\section{Вложенные стягивающиеся отрезки с действительными концами}
\subsection{Определения.}
\begin{tcolorbox}
    $x,y \in R,\; x < y$ для ограниченного отрезка\\
    \textbf{отрезок} $[x, y] = \{z \in Z; x \leq z \leq y\}$ - \\
    \textbf{полуинтервал} $[x,y) = \{ z \in Z ; x\leq z < y \}$\\
    \textbf{полуинтервал} $(x,y] = \{ z \in Z ; x < z \leq y \}$\\
    \textbf{интервал} $(x,y) = \{ z \in Z; x < z < y \}$\\
    \textbf{Определение}\\
    Если $z \in (x, y)$, то $U_z$ $(x,y)$ - \textbf{окрестность} точки (числа) $z$\\
    \textbf{Определение}
    Диаметр $U_z = y-x$\\
    $\forall \eps > 0: (x-\eps,x+\eps)$ - $U_{x,\eps}$ - \textbf{$\eps$-окрестность} числа x.
\end{tcolorbox}

\subsection{Окрестностное определение предела последовательности.}
\begin{tcolorbox}
    \textbf{Окрестностное определение предела $\{x_n\}$.}
    $x_n \xrightarrow[n\rightarrow \infty]{} x,\;\; (\forall x_n, x\in\R)$, если\\
    $\forall U_x\; \exists n_{U_x}: \forall n > n_{U_x} \;\; x_n \in U_x $\\
    \textbf{Определение предела $\{x_n\}$.}
    $ x_n\xrightarrow[n\to\infty]{} x$, если\\
    $\forall \eps > 0\; \exists n_\eps: \forall n > n_\eps \;\; |x_n-x| < \eps$
\end{tcolorbox}

\begin{tcolorbox}
    \textbf{Определение}
    $\{[a, b]\}$ - вложенные стягивающиеся отрезки с R концами
\end{tcolorbox}

\begin{tcolorbox}
    \textbf{Последовательность стягивающихся вложенные отрезков с действительными концами}
    \[\{[x_n, y_n]\}, \forall n\;x_n,y_n \in \R\]
    \begin{enumerate}
        \item $\forall n [x_{n+1},y_{n+1}]\subset[x_n,y_n]$
        \item $y_n-x_n \to 0$
    \end{enumerate}  
\end{tcolorbox}

\begin{tcolorbox}
    \textbf{Утверждение}
    Центральный класс всегда не пуст в R\\
    Вариант с записи: \\
    $\exists!z\in\R: \forall n\; z\in [x_n, y_n]$
\end{tcolorbox}

\begin{tcolorbox}[title=Доказательство., breakable]
    \includegraphics[width=0.5\linewidth]{Док-во утверждения об отрезках.png}\\
    $\{ [a_n,b_n] \} \rightarrow \{ [\alpha_n,\beta_n] \}$, где $a_n,b_n \in R\;\;\; \alpha_n\beta_n \in Q$\\
    $|\beta_n - \alpha_n| = |(\beta_n - b_n)+(b)n-a_n)+(a_n-\alpha_n)| \leq$ - неравенство треугольника\\ 
    $|\beta_n-b_n|'<1/n'+|b_n-a_n|'\rightarrow 0' +|a_n-\alpha_n| '<1/n' \rightarrow 0$\\
    $\alpha_1,\ $\\

    С записи:\\
    \begin{enumerate}
        \item Множество $\Q$ всюду плотно, то есть $\forall (x, y)\; \exists r\in \Q, r\in (x, y)$. $x < y$ по определению $\exists r \in Q$\\ 
        \includegraphics[width=0.5\linewidth]{images/Вложенные отрезки/Доказательство плотность множества Q.png}
	\begin{enumerate}
		\item  $\beta \in \Q$\\
		Здесь могла бы быть ваша картинка.
	        \item $ x_2 - \frac{1}{2^2}$ 
	\[ \alpha_2 \in (\alpha_1; x_2) \;  \alpha_2 \in (x_2 - \frac{1}{2^2}; x_2) \]
	\[ \beta_1 \in (y_2, \beta_1) \cap (y_2; y_2 + \frac{1}{2^2}) \] 
	\[ [\alpha_1, \beta_1] \supset [x_1, y_1] \]
	\[ [\alpha_1, \beta_1] \supset [\alpha_2, \beta21] \supset [x_2, y_2] \]
	Вложенные отрезки с рациональными концами.
	\[ \{ [ \alpha_n, \beta_n ] \} \]
	Эта последовательнеость стягивается. $ \forall n\; [x_n, y_n] \subset [\alpha_n, \beta_n] $\\
	\[ \beta_n - \alpha_n = (\beta_n - y_n) + (y_n - x_n) + (x_n - \alpha_n) \]
         $ \exists! r \in \R $ - общая точка всех $[\alpha_n, \beta_n]$\\
	 Пусть $\exists n \; z < x_n$\\
	 В ведь здесь тоже мог бы быть ваш рисунок.
	\end{enumerate}
\end{enumerate}
\end{tcolorbox}

\begin{tcolorbox}
    \textbf{Принцип полноты множества по Вейерштрассу}
    Принцип полноты множества по Вейерштрассу означает, что любое ограниченное сверху множество имеет точную верхнюю грань     
\end{tcolorbox}

\begin{tcolorbox}
    \textbf{Определение}
    $\{a_n\}$ монотонна, если возрастает/ строго возрастает/убывает/строго убывает.
\end{tcolorbox}

\begin{tcolorbox}
    \textbf{Теорема по Вейерштрассу}
    \begin{enumerate}
        \item $\{a_n\} \uparrow \; \Rightarrow a_n \to sup\{ a_n \}$
        \item $\{a_n\} \downarrow \; \Rightarrow a_n \to inf\{ a_n \}$
    \end{enumerate}
\end{tcolorbox}

\begin{tcolorbox}
    \textbf{Определение}
    $sup \{a_n\}$ - это $sup$ множества членов последовательности.
\end{tcolorbox}

\begin{tcolorbox}[title=Доказательство полноты $\R$ по Вейерштрассу]
    \textbf{Доказательство}
    \begin{enumerate}
        \item Ограничено сверху $\exists M: \forall n a_n \leq M \Rightarrow \exists sup\{a_n\} = M \in \R$\\
        По определению предела $\forall \varepsilon > 0 \exists n_\varepsilon: \forall n > n_\varepsilon |a_n - M| < \varepsilon $\\
        $\Leftrightarrow M-\varepsilon<a_n<M+\varepsilon$ т.к. $M=sup\{a_n\}$, то $a_n \leq M$ надо проверить $M-\varepsilon<a_n\leq M$\\
        M - наименьшая верхняя грань $\Rightarrow \exists n_\varepsilon: \forall n > n_\varepsilon \;\;\;\;\; M-\varepsilon \leq a_{n_\varepsilon} \leq a_n \leq M $\\
        т.е. по определению 
        \[M = \lim_{n\to\infty} a_n\]
        \textbf{Следствие} $\{a_n\}$ - монотонна $\{a_n\}$ - сходится $\Leftrightarrow \{a_n\}$ - ограничена.
    \end{enumerate}
\end{tcolorbox}

\section{\S 3 Число e как предел последовательности.}
Вспомним неравенство среднего геометрического и среднего арифметического.
\[\forall k: a_k>0\;\;\;\ \sqrt[n]{a_1\cdot a_2\cdot ... \cdot a_n} \leq \cfrac{a_1+a_2+...+a_3}{n} \]
Пусть $a_1 = a > 0,\;a_2=a_3=...=a_{n+1}=d>0$\\
$\sqrt[n+1]{a\cdot b^n} \leq \cfrac{a+nb}{n+1}$\\
$(1+\cfrac{1}{n})^k$ - сходится
\begin{tcolorbox}[title=Доказательства монотонности]
    \begin{enumerate}
        \item $(1+\cfrac{1}{n})^k \uparrow$\\
        Пусть $a = 1,\;b=1+\cfrac{1}{n}$\\
        $\sqrt[n+1]{a\cdot(1+\cfrac{1}{n})^n}\; < \cfrac{1+n(1+\cfrac{1}{n})}{n+1} = \cfrac{n+1}{n+1} = 1+\cfrac{1}{n+1} \uparrow n+1 $
        \begin{tcolorbox}
            \[ (1+\cfrac{1}{n})^n < (1+\cfrac{1}{n+1})^{n+1} \]
        \end{tcolorbox}
        \item $(1-\cfrac{1}{n})^k \uparrow$\\
        Пусть $a = 1,\;b=1-\cfrac{1}{n}$\\
        $\sqrt[n+1]{a\cdot(1-\cfrac{1}{n})^n}\; < \cfrac{1+n(1-\cfrac{1}{n})}{n+1} = \cfrac{n+1-1}{n+1} = 1-\cfrac{1}{n+1} \uparrow n+1 $
        \begin{tcolorbox}
            \[ (1-\cfrac{1}{n})^n < (1-\cfrac{1}{n+1})^{n+1} \]    
        \end{tcolorbox}
        \item $(1+\cfrac{1}{n})^{n+1} = (\cfrac{1}{\frac{n}{n+1}})^{n+1} = \cfrac{1}{(1-\frac{1}{n+1})^{n+1} \downarrow } $
    \end{enumerate}
\end{tcolorbox}
$\forall k,m \;\;\; n=max\{k,m\}$\\
\[ (1+\cfrac{1}{k})^k \leq (1+\cfrac{1}{n})^n < (1+\cfrac{1}{n})^{n+1} \leq (1+\cfrac{1}{m})^{m+1} \Rightarrow \]
$(1+\cfrac{1}{n})^n$ - ограничена сверху $\Rightarrow$ сходится.

\begin{tcolorbox}
    \textbf{Определение} 
    \[e = \lim_{n\to \infty} (1+\cfrac{1}{n})^n \]
\end{tcolorbox}

\begin{tcolorbox}[title=Доказать/подумать]
    \begin{enumerate}
        \item \[ e = \lim_{n\to\infty} (1+\cfrac{1}{1!} +\cfrac{1}{2!} + ... + +\cfrac{1}{n!}) \]
        \item $(1+\cfrac{1}{n})^{n+2}$
        \item Сколько слагаемых нужно взять, чтобы получить e с точностью $10^{-3}$\\
        \item Какое нужно взять n, чтобы получить e с точностью $10^{-3}$
    \end{enumerate}
\end{tcolorbox}

\section{\S4 Последовательности.}
\begin{tcolorbox}
    \textbf{Определения:} предела, последовательности, сходящейся последовательности, ограниченной последовательности.
\end{tcolorbox}

\begin{tcolorbox}[title=Теоремы ]
    \begin{enumerate}
        \item Единственность предела
        \item Ограниченность сходящейся последовательности
        \item $\{a_n\} \to a \Leftrightarrow a_n = a + \alpha_n,\; \alpha_n$ - б.м.
        \item Арифметические свойства
    \end{enumerate}
\end{tcolorbox}

\begin{tcolorbox}
    \textbf{Теорема 1.} Если $\exists N: \forall n > N \;\; a_n = a \Rightarrow a_n\to a$
\end{tcolorbox}
\begin{tcolorbox}[title=Доказательство]
    \[ \forall \varepsilon>0 \;\;\forall n>N \;\; |a_n - a| = 0 < \varepsilon \Rightarrow a_n\to a\] по определению 
\end{tcolorbox}

\begin{tcolorbox}
    \textbf{Теорема 2.} Если $ a_n\to a,\;\; b_n\to b \;\;\;\forall n > N \;\; a_n \leq b_n \Rightarrow a\leq b $
\end{tcolorbox}
\begin{tcolorbox}[title=Доказательство]
    \textbf{От противного:}\\ 
    Пусть $ a < b$
\end{tcolorbox}
\textbf{НАДО ПОСМОТРЕТЬ ПРЕДЫДУЩИЕ ЛЕКЦИИ. ЭТО ОТТУДА.}

\section{\S 5 Теорема Больцано-Вейерштрасса}
$n_k$ - строго возрастает последовательность натуральных чисел $\Rightarrow n_k\geq k$ .

\begin{tcolorbox}
    \textbf{Определение} $\forall \{a_n\} \; \{a_n\}$ - подпоследовательность $\{a_n\}$
\end{tcolorbox}
$\{a_n\} = 2^n$, тогда 2, 16, 64, 128 - подпоследовательность\\
$\begin{cases}
    2, \textbf{3}, 4, 16, ...\\
    16, \textbf{2}, 32, ...     
\end{cases}$ - не подпоследовательности\\

\begin{tcolorbox}[title=Теорема Больцано-Вейерштрасса]
    $ \forall $ ограниченой $\{a_n\} \exists $сходящаяся подпоследовательность $ \{a_n\} $
\end{tcolorbox}

\begin{figure}[h]
    \centering
    \includegraphics[width=0.5\linewidth]{Т Б-В.png}
\end{figure}

\begin{tcolorbox}[title=Доказательство]
    Прицип вложенных отрезков.\\
    $\exists [c,d]: \forall n\; a_n\in[c,d]$
    \begin{enumerate}
        \item $[c_1,d_2] = [c,d] \;\; \forall a_{n_1} \in [c_2,d_1]$
        \item $b_1 = \cfrac{c_1+d_1}{2}$
                $[c_2,d_2]$ тот из $[c_1,b_1]$ и $[b_1,d_1]$ на котором содержится бесконечно много членов последовательности $\{a_n\}$\\
        $a_{n_2}: a_{n_2} \in [c_2,d_2], n_2>n_1$
        \item $b_2 = \cfrac{c_2+d_2}{2}$\\
        $[c_3,d_3]$ тот из $[c_2,b_2]$ и $[b_2,d_2] ...$\\
        $a_{n_3}: a_{n_3} \in [c_3,d_3], n_3>n_3$

        $\{a_n\} \;\;a_{n_k} \in [c_k,d_k], n_k > n_{k-1}$
        $[c_1,d_1] \supset [c_2,d_2] \supset ... \supset [c_k,d_k] \supset ...$\\
        длина $[c_k,d_k] = \cfrac{1}{2^{k-1}}$
        длина $[c_1,d_1] \to 0$
        $\Leftrightarrow \exists a` = U[c_k,d_k]$\\
        $\forall k\;\; |a_n-a`| \leq \cfrac{1}{2^{k-1}}$ - длина $[c_1,d_1] \to 0$
        $a_{n_k}, a`\in [c_k,d_k] \Rightarrow a_{n_k} \to 0`$
    \end{enumerate}
\end{tcolorbox}

\begin{tcolorbox}
    \textbf{Определение} Частичный предел $\{a_n\}$ - предел $\forall$ сходящейся подпоследовательности $\{a_n\}$
\end{tcolorbox}
\begin{tcolorbox}[title=Следствие из теоремы]
    $\forall \{a_n\} \exists$ подпоследовательность $\{a_{n_k}\}$ которая имеет либо конечное либо бесконечное число пределов.
\end{tcolorbox}
\begin{tcolorbox}[title=Доказательство]
    Если в $\{a_n\} \exists \{a_{n_k}\} a_{n_k} \to a` \in \R$\\
    Если такого нет, то по Теореме Б-В $\{a_{n}\}$ - не ограничена сверху или снизу.\\

    Если $\{a_{n}\} $ - не ограничена сверху:
    \begin{enumerate}
        \item 1 - не верхняя грань $\{a_{n}\}: a_{n_1}: a_{n_1} > 1$
        \item 2 - не верхняя грань $\Rightarrow \exists n_2: a_{n_2}>2, n_2>n_1$
        \item[...]
        \item [k.] $\exists n_k: a_{n_k}>max\{k, a_1, a_2, ..., a_{n_{k-1}}\}, n_k>n_{k-1}$
    \end{enumerate}
\end{tcolorbox}

\section{\S 5. Критерий Коши сходящейся последовательности.}
\subsection{Фундоментальные последовательности.}
\begin{tcolorbox}
    \textbf{Последовательность $\{a_n\} $фундаментальна} - если
    \[\forall \eps>0 \;\; \exists n_\eps : \forall n, m > n_\eps \;\; |a_n-a_m| < \eps\]
    \[ \forall \eps>0 \;\; \exists n_\eps : \forall n > n_\eps,\forall  p \;\; |a_{n+p}-a_n| < \eps \]
\end{tcolorbox}

\subsection{Критерий Коши.}
\begin{tcolorbox}
    $\{a_n\}$ - сходится $\Leftrightarrow \{a_n\}$ - фундаментальна.
\end{tcolorbox}

\begin{tcolorbox}[title=Доказательство]
    Взять с записи
\end{tcolorbox}

\section{\S 6. Верхний и нижний предел.}
\textbf{Утверждение 1.} $\forall$ последовательность имеет хотя-бы 1 частичный предел (конечный или бесконечный).
\begin{tcolorbox}[title=Доказательство]
    \begin{enumerate}
        \item Пусть $\{a_n\}$ - не ограничена сверху.\\
        Напоминание: $\{ a_n \}$ - не ограничена сверху, если $\forall M\; \exists a_n > M \Leftrightarrow \exists$ бесконечно много таких членов.\\
        $a_n > 1, a_{n_2} > 2,\ n_2>n_1$\\
        $a_{n_k} > k, n_k > n_{k-1}$\\
        $\{a_{n_k}\} \to +\infty$
        \item Если $\{a_n\}$ не ограничена снизу, то $\exists a_{n_k}\to -\infty$ (аналогично)
        \item Если $\{a_n\}$ - огрничена, то Б-В.
        
    \end{enumerate}
\end{tcolorbox}

\textbf{Утверждение 2.} критерий частичного предела.\\
a - частичный предел $\{a_n\} \Leftrightarrow \forall U_a$ принадлежит б.м. членов последовательности.
\[ \forall U_a \exists a_{n_k}\in U_a^\circ = U_a /\{a\} \] - ЭТО ХУЙНЯ. НАДО НАЙТИ ОШИБКУ\\
На экзамене: что-то может быть.\\

\begin{tcolorbox}[]
    $\forall U_a \; \exists k_{U_a}: \forall k > k_{U_a} \;\; a_{n_k} \in U_a$\\
    $\Rightarrow$ в $U_a$ бесконечно много членов последовательности.\\
    $\Leftarrow$ в $\forall U_a$ бесконечно много членов $\{a_n\}$.\\
    $\eps_1 = \cfrac{1}{2} \;\; U_a = (a-\eps; a+\eps)$ - отсюда любой член $\{a_n\}$\\
    $\eps_2 = \cfrac{1}{2^2} \;\; \exists a_{n_2} \in (a-\eps_2; a+\eps_2), n_2>n_1$\\
    ...\\
    $\eps_k = \cfrac{1}{2^k} \;\; \exists a_{n_k} \in (a-\eps_k; a+\eps_k), n_k>n_{k-1}$\\
    $\{a_{n_k}\} a_{n_k} \to a,$ при $k\to \infty$\\
    $\forall k\;\;\; a-\cfrac{1}{2^n} \to a= a - \eps_k < a_{n_k} < a_\eps = a+ \cfrac{1}{2^k} \to a$\\
    По теореме о зажатой последовательности $a_{n_k} \to \infty$
\end{tcolorbox}

\begin{tcolorbox}
    \textbf{Определене.} \\
    Наибольший из часичных пределов. $\{a_n\}$ - верхний прдел $a_n$\\
    Наименьший из часичных пределов. $\{a_n\}$ - нижний прдел $a_n$
\end{tcolorbox}

\textbf{Теорема} $\forall \{a_n\} \;\; \exists$ верхний и нижний предел.
\begin{tcolorbox}
    \begin{enumerate}
        \item $\exists$ $\underline{\lim} a_n$ - нижний предел $a_n$\\
        Пусть $\{a_n\}$ - не ограничена те $\forall M \exists a_n < M$. Таких $a_n$ - бесконечно много.\\
        $a_{n_1} < -1, a_{n_2} < min\{ -2, a_1, ..., a_{n_1} \} - 1$
        $n_2 > n_1$\\
        $a_{n_k} < min\{-k, a_1, ..., a_{n_{k-1}}\} - 1$\\
        $a_{n_k} - k, n_k > n_{k-1}$\\
        $\{a_{n_k}\}$ - подпоследовательность.\\
        \[ a_{n_k} < -k \Rightarrow a_{n_k} \to -\infty \]
        \item Пусть $\{ a_n \}$ - ограничена снизу.
            \item[a)] $\{a_n\}$  имеет конечные частичные пределы.\\
            A - множество конечных частных пределов. $A \not= \varnothing$ и ограничена снизу.\\
            $\exists inf\; A = a$. Покажем, что $a = \underline{\lim}_{n\to\infty} a_n$
            \begin{center}
                \includegraphics[width=0.2\linewidth]{images/Пределы.png}
            \end{center}
            $ \forall \eps > 0 \exists a` \in A $\\
            $a \leq a` < a+\eps$
            \item[б)] $\{a_n\}$ нет конечных частичных пределов.\\
            $a_n \to +\infty$\\
            Если $a_n \not\to +\infty \;\; \exists M: \forall N \;\; \exists n > N\;\;\; a _n\leq M$ \\
            $\exists$ бесконечно много $\{a_n\} < M$ по Б-В $\exists$ конечный частичный предел.
    !!! Каждое действитеьное число является её частиыным пределом.
    \end{enumerate}
\end{tcolorbox}

\section{Числовые ряды}
\subsection{Определения и элементарные факты.}
$ \{a_k\} $ - ЧП
\[ \{a_k\} \to S_n = \sum^n_{k=1}a_k \]
\subsubsection{Определение}
\begin{tcolorbox}
    \textbf{Определяем бесконечную сумму}.\\
    $\sum^\infty_{k=1} a_k$ - \textbf{ряд}\\
    $a_k$ - \textbf{элемент ряда (общий член)}.\\
    $S_n$ - \textbf{n-ая частичная сумма ряда}.\\
    Если $S_n$ - сходится, то ряд ($\sum^\infty_{1} a_k$) называется \textbf{сходящимся}, $S:=\lim_{n\to\infty} s_n$ - \textbf{суммой ряда}: $\sum_1^\infty = S$.\\
    Если $\{S_n\}$ расходится, то $\sum^\infty_{1} a_k$ - \textbf{расходится}.
\end{tcolorbox}
\subsubsection{\textbf{Теорема 1.}}
\[ \sum_1^\infty a_k \pm \sum_1^\infty b_k = \sum_1^\infty (a_k \pm b_k) \]
\subsubsection{\textbf{Теорема 2. Критерий Коши о сходимости ряда.}}
$ \sum_1^\infty a_k$ - сходится $\Leftrightarrow \forall \eps > 0\;\; \exists n_\eps : \forall n > n\eps \forall p \;\;\; |S_{n+p} - S_n| < \eps $
\[ |\sum_1^{n+p} a_k - \sum_1^n a_k| = |\sum_{n+1}^{n+p} a_k| < \eps \]
\subsubsection{\textbf{Следствие 1. Изменение конечного числа членов ряда не влияет на сходимость.}}
Сумма конечно может измениться, но на сходимость это не влияет.
\subsection{\textbf{Следствие 2. Необходимое условие сходимости ряда.}}
Можно считать определением: Если $\sum_1^\infty a_k$ сходится $\Rightarrow a_k \xrightarrow[k\to\infty]{} 0$\\
\begin{tcolorbox}[title=Доказательство]
    $\sum_1^\infty a_k$ - сходится $\forall \eps > 0\; \exists n_\eps:\; \forall n > n_\eps\;\; \forall p \ |\sum_{n+1}^{n+p} a_k| < \eps \Leftrightarrow |a_{n+1}| < \eps$\\
    При p = 1, то есть $\{a_n\}$ - б.м. по определению.
\end{tcolorbox}
\subsubsection*{Пример 1.}
Если |q| < 1
\[ \sum_1^\infty q^k, S_n = 1+...+q^n = \cfrac{1-q^{n+1}}{1-q} \xrightarrow[n\to\infty]{} \cfrac{q}{1-q} \]
\subsubsection*{Пример 2.}
$\sum_1^\infty \cfrac{1}{n}$\\
a, b, c положительные. c - среднее гармоническое a и b, если $\cfrac{2}{c} = \cfrac{1}{a} + \cfrac{1}{b}$\\
$a = \cfrac{1}{n-1}, b = \cfrac{1}{n+1}$\\
$\cfrac{2}{c} = (n - 1) + (n + 1) = 2n$\\
Каждый элемент является средним гармоническим 2-х соседий. По критерию Коши ряд - расходящийся.
\[ |\sum_{n+1}^{n+p} \cfrac{1}{k}| = \cfrac{2}{n+1} + ... + \cfrac{1}{n+p} \geq \cfrac{p}{n+p}\]
Пусть p = n. $\cfrac{p}{n+p} = \cfrac{1}{2}$.
\subsubsection*{\textbf{Пример 3.}}
\[ 1 - 1 + \cfrac{1}{2} - \cfrac{1}{2} + \cfrac{1}{3} - \cfrac{1}{3} + ...\]
$S_{2n} = 0$\\
$S_{2n+1} = \cfrac{1}{n+1} \to 0$\\
А теперь давайте мухлевать. Переставим сумму ряда. Шоу ИМПРОВИЗАЦИЯ!!!\\
$(1+\cfrac{1}{2}) - 1 + (\cfrac{1}{3} + \cfrac{1}{4} +\cfrac{1}{5} + \cfrac{1}{6} + \cfrac{1}{7} + \cfrac{1}{8} + ... + \cfrac{1}{11}) - \cfrac{1}{2}$\\
Берём много положительных слагаемых и вычитаем меньшее по модулю число. Из-за этого ряд расходится.
\subsubsection*{\textbf{Пример 4.}}
\[ \sum^\infty_{0} (-1)^k = (1-1) + (1-1) + ... \]
\[ \sum^\infty_{0} (-1)^k = 1 - (1-1) - (1-1) - ... \]

\subsection{Абсолютно сходящийся ряд.}
\begin{tcolorbox}
    \textbf{Определение.} Если $\sum^\infty_{1} |a_k|$ сходится, то $\sum^\infty_{1} a_k$ сходится абсолютно.
\end{tcolorbox}
\subsubsection{\textbf{Теорема 1. Если ряд сходится абсолютно, то ряд сходится.}}
\begin{tcolorbox}[title=Доказательство (Критерий Коши)]
    По критерию Коши, т.к. $\sum^\infty_{1} |a_k|$ сходится, то $\forall \eps > 0\; \exists n_\eps:\; \forall n > n_\eps\; \forall p\;\; |\sum^{n+p}_{n+1} a_k| \leq |\sum^{n+p}_{n+1} |a_k|| < \eps \Rightarrow$ по критерию Коши $\sum a_k$ сходится.
\end{tcolorbox}
\subsubsection*{\textbf{Пример 1.}}
$1 - 1 + \cfrac{1}{2} - \cfrac{1}{2} + ... $ - сходится, но не абсолютно.
\subsubsection{\textbf{Определение.}} 
Если $\sum^{\infty}_{1} a_k$ сходится, а $\sum^{\infty}_{1} |a_k|$ расходится, то $\sum a_k$ сходится условно.
\subsubsection{\textbf{Теорема 2.}}
$\sum^{\infty}_{1} a_k, \forall k\; a_k \geq 0$\\
$\sum^{\infty}_{1} a_k$ сходится $\Leftrightarrow \{S_N\}$ ограничена.
\begin{tcolorbox}[title=Доказательство]
    \[\forall n\; S_{n+1} = \sum^{n+1}_{1} a_k \geq \sum^{n}_{1} a_k = S_n \]
    $S_n \uparrow$ возрастающая $\{S_n\}$ сходится $\Leftrightarrow$ ограничена.
\end{tcolorbox}
\subsubsection{Теорема 3. Признак сравнения.}
Для комплов не годится.\\
$\sum^{\infty}_{1} a_k,\;\; \sum^{\infty}_{1} b_k,\;\; \forall k\;\; a_k \geq \underline{b_k \geq 0}$\\
Тогда:
\begin{enumerate}
    \item Если $\sum^{\infty}_{1} a_k$ сходится $\Rightarrow \sum^{\infty}_{1} b_k$ сходится.
    \item Если $\sum^{\infty}_{1} a_k$ расходится $\Rightarrow \sum^{\infty}_{1} b_k$ расходится.
\end{enumerate}
\begin{tcolorbox}[title=Доказательство]
Следствие критерия сходимости ряда с неотрицательными членами.\\
\begin{enumerate}
    \item $A_n = \sum^{n}_{1} a_k,\;\; B_n = \sum^{n}_{1} b_k $\\
    $A_n \uparrow,\; B_n \uparrow$ и $A_n \geq B_n$ ($A_n$ можарирует $B_n$)\\
    Если $\sum a_k$ сходится $\Rightarrow \{A_n\}$ ограничена сверху $\Rightarrow \{B_n\}$ ограничена сверху $\Rightarrow \sum b_k$ сходится.
\end{enumerate}
\end{tcolorbox}

\subsubsection{Следствие. (Признак сравнения).}
$\sum a_k,\; \sum b_k \;\; \forall k\; a_k \geq |b_k| > 0$. Не отрицательность $a_k$.\\
Сходимость $\sum a_k \Rightarrow$ сходимость $\sum b_k$ (абсолютная).
\subsubsection*{Пример 1.}
\[\sum^{\infty}_{1} \cfrac{\sin n}{n^2}\]
\[|\cfrac{\sin n}{n^2}| \leq \cfrac{1}{n^2} < \cfrac{1}{(n-1)n}, n \not= 1\]
\[ \cfrac{1}{(n-1)n} = \cfrac{1}{n-1} - \cfrac{1}{n} \]
\[ \sum^{\infty}_{2} \cfrac{1}{n-1)n} = \sum^{\infty}_{2}( \cfrac{1}{-1} - \cfrac{1}{n}) \]
\[ S_n = \sum^{n}_{2} (\cfrac{1}{k-1} - \cfrac{1}{k}) = (\cfrac{1}{2-1} - \cfrac{1}{2}) + (\cfrac{1}{2} - \cfrac{1}{3}) + ... \]

\subsubsection{Теорема 4. Признак Коши.}
\[ \sum_1  a_k\]
\[q = \overline{lim} \sqrt[k]{|a_k|} \]
Тогда:
\begin{enumerate}
    \item $q < 1 \Rightarrow$ абсолютно сходится.
    \item q > 1 расходится
    \item q = 1 ?
\end{enumerate}
\begin{tcolorbox}[title=Доказательство]
    Сравнение с геометрической прогрессией.\\
    \begin{enumerate}
        \item $ 0 \leq q < p < 1 $
        \includegraphics[width=0.5\linewidth]{images/Ряды/Признак Коши Док-во.png}\\
        $\exists n_p: \forall n>n_p$\\
        $\sqrt[k]{a_k} < p \uparrow \Leftrightarrow |a_k| < p_k$\\
        $\sum^{\infty}_{n+1} p^k$ геометрическая прогрессия с положительным знаком |x|.\\
        $\Rightarrow \sum^{\infty}_{1} a_k$ сходится абсолютно (по признаку сравнения)
        \item q = $\overline{lim} \sqrt[k]{a_k} > 1$\\
        $1 > p > 1$\\
        $\forall$ окрестности $q$ бесконечно много членов $\sqrt[k]{|a_k|} $
    \end{enumerate}
\end{tcolorbox}
\textbf{Замечание.} Признак Коши бесполезно использовать, если ряд не похож на геометрическую прогрессию.\\
\textbf{Запомните.} Признак коши достаточное условие абсолютной сходимости.\\
\textbf{Следствие} Если $ \sqrt[n]{|a_n|} \leq q < 1 \Rightarrow \exists q,N\; \forall n > N: \;  \sum a_n$ сходится абсолютно

\textbf{Пример.} $ \sum^\infty_{n=1} (2+(-1)^n)^n \cdot z^n$\\
\[
	\sqrt[n]{((2+(-1)^n)^n z^n} = (2+(-1)^n)|z|\\
	n = 2k: \sqrt[n]{|a_n|} = 3|z| = \overline{\lim} \sqrt[n]{|a_n|} \\
	n = 2k+1: \sqrt[n]{|a_n|} = |z|\\ 
\]
Если: 
	$3|z| < 1$ сходится $|z| < \cfrac{1}{3}$\\
	$3|z| > 1$ расходится $|z| > \cfrac{1}{3}$\\
	$3|z| = 1 |z| = \cfrac{1}{3} $

\textbf{Теорема 5. Признак д-Аламбера}\\
!Он всегда слабее признака Коши.
\begin{tcolorbox}
	\[
		\sum_{n=1}^{\infty}: |\cfrac{a_{n+1}}{a_{n}}| \to q
	\]
\begin{enumerate}
	\item $q < 1 \Rightarrow$ абсолютно сходится
	\item $q < 1 \Rightarrow$ расходится
	\item $q = 1 \Rightarrow$ Ничего не даёт
\end{enumerate}
\end{tcolorbox}

\begin{tcolorbox}[title=Доказательство.]
	\begin{enumerate}
		\item  Сравнение с геометрической прогрессией.\\
		$\exists n_p: \forall n > n_p \;\; |\cfrac{a_{n + 1}}{a_n}| < p$\\
		Пусть для $\forall n$\\
		$ a_{n+1} = \cfrac{a_{n+1}}{a_n} \cdot  \cfrac{a_{n}}{a_{n-1}} \cdot ... \cdot \cfrac{a_{2}}{a_1} $
		$ |a_{n+1}| = |\cfrac{a_{n+1}}{a_n}| \cdot  |\cfrac{a_{n}}{a_{n-1}}| \cdot ... \cdot |\cfrac{a_{2}}{a_1}| \cdot |a_1| < p^n |a_1| $\\
		$ \displaystyle\sum_{}^{} p^n |a_1|$ сходится (геометрическая прогрессия)
	\item $\exists N: |\cfrac{a{_n+1}}{a_n}| > 1\;\; \forall n > N$\\
		Пусть $ \forall N\; |\cfrac{a_{n+1}}{a_n}| > 1 $\\
		$ |a_{n+1}| > |a_{n}| > |a_{n-1}| > ... |a_{1}| > 0 \Rightarrow a_n \not\to 0$
	\item $ \displaystyle\sum_{1}^{\infty} \cfrac{1}{n}\;\; |\cfrac{a_{n+1}}{n}| = \cfrac{n}{n+1} \to 1$\\
		$ \displaystyle\sum_{1}^{\infty} \cfrac{1}{n^2}\;\; |\cfrac{a_{n+1}}{n}| = \cfrac{n^2}{(n+1)^2} \to 1$\\
 	\end{enumerate}
\end{tcolorbox}

\textbf{Следствие.} $\forall n > N:\; |\cfrac{a_{n+1}}{a_n}| \leq q < 1 \Rightarrow \sum a_n$ сходится абсолютно.

\begin{tcolorbox}[title=Доказательство следствия]
	Пусть $ \forall n\; |\cfrac{a_{n+1}}{a_n}| \leq q < 1 $\\
$ |a_{n+1}| = |\cfrac{a_{n+1}}{a_n} \cdot |\cfrac{a_{n}}{a_{n-1}} \cdot ... \cdot |\cfrac{a_{2}}{a_1} \cdot |a_1| \leq q^n |a_1|$ Т.к. $|q| < 1$, то $\displaystyle\sum_{}^{} q^n|a_1|$ сходящаяся геометрическая прогрессия.
\end{tcolorbox}
\textbf{Вопрос на 5:} Если можно исследовать по Деламберу то можно и по Коши.\\

\textbf{Пример}
	$\displaystyle\sum_{1}^{\infty} \cfrac{1}{(3+)-1)^n)^n}$\\
	$\sqrt[n]{|a_n|} = \cfrac{1}{3+(-1)^n} = \cfrac{1}{4}, n = 2k$\\
	$\sqrt[n]{|a_n|} = \cfrac{1}{3+(-1)^n} = \cfrac{1}{2}, n = 2k+1$\\
	$\sqrt[n]{|a_n|} \leq \cfrac{1}{2} < 1$

	$|\cfrac{a_{n+1}}{a_{n}} = \cfrac{\cfrac{1}{2^{2k+1}}{\cfrac{1}{4^{2k}}}} = \dots$\\
\textbf{Пример} 
    $\sum_{1}^{\infty} \cfrac{1}{(3+)-1)^n)^n}$\\
	$\sqrt[n]{|a_n|} = \cfrac{1}{3+(-1)^n} = \cfrac{1}{4}, n = 2k$\\
	$\sqrt[n]{|a_n|} = \cfrac{1}{3+(-1)^n} = \cfrac{1}{2}, n = 2k+1$\\
	$sqrt[n]{|a_n|} \leq \cfrac{1}{2} < 1$

$
	|\cfrac{a_{n+1}}{a_{n}} = \cfrac{\cfrac{1}{2^{2k+1}}}{\cfrac{1}{4^{2k}}} = \cfrac{4^{2k}}{2^{2n+1}} = \cfrac{1}{2}2^{2k}
    n = 2k
$

\[
	|\cfrac{a_{n+1}}{a_{n}}| = \cfrac{\cfrac{1}{2^{2k+1}}}{\cfrac{1}{4^{2k}}} = \cfrac{4^{2k}}{2^{2n+1}} = \cfrac{1}{2^{n+1}}\\
    n = 2k+1
\]

\textbf{Замечание.} $|\cfrac{a_{n+1}}{a_n}| \leq 1 < 1$, а не $|\cfrac{a_{n+1}}{a_n}| < 1$

\textbf{Пример.} \[ \sum_0^\infty \cfrac{z^n}{n!} \]
\[ |\cfrac{a_{n+1}}{a_n}| = \cfrac{\cfrac{z^{n+1}}{(n+1)!}}{\cfrac{z^n}{n!}} = \cfrac{|z|}{n+1} \to 0 \]

\textbf{Теорема 6. Теорема Коши о сходимости монотонных рядов. }
\[\sum^{\infty}_{1} a_n,\ a_n \downarrow,\ \forall n\; a_n \geq 0\]
$ \sum^{\infty}_{1} a_n$ сходится $\Leftrightarrow \sum^{\infty}_{1} 2^n \cdot a_{2n}$ сходится

\begin{tcolorbox}
    Если $ \sum^{}_{} a_k:\ \forall k\; a_k \geq 0$, то $ \sum^{}_{} a_k$ сходится $\Leftrightarrow S_n=$ $ \sum^{n}_{1} a_k$ ограничена\\
    $ a_2 \leq a_2 \leq a_1 $\\
    $2a_4 < a_3 + a_4 \leq 2a_2 $\\
    $\cfrac{1}{2} 2^3 a_{2^3} = 2^2a_{2^3} = 2^2a_8 \leq a_5+a_6+a_7+a_8 \leq 4a_4 = 2^2 a_{2^2}$\\
    $\cfrac{1}{2}2^{k+1} a_{2^{2k+1}} = 2^ka_{2^{k+1}} \leq a_{2^k+1} + ... + a_{2^{k+1}} \leq 2^ka_{2^k}$
    $\sum^{2^{k+1}}_{n=2} a_n \leq \sum^{2^{k}}_{n=0} 2^n a_{2^n}$
\end{tcolorbox}

\section{Галава 3. Функции}
\subsection{Понятие функции.}

\begin{tcolorbox}
	\textbf{Определени.} $X$ и $Y$ множества\\
	\[ \forall x \in X \xrightarrow[f(x)]{} y \in Y \]
	$(f, X, Y) $ - функция.
\end{tcolorbox}

\begin{tcolorbox}
	\textbf{Определени.} $f(X) := \{ y \in Y: \exists x \in X\;\ f(x) = y \}$ - образ X при отображении $y = f(x)$
\end{tcolorbox}
\subsubsection*{Примеры:}
\begin{enumerate}
	\item  $y = x^2,\; x \in [0; +\infty)$
	\item $y = x^2,\; x \in (-\infty; +\infty)$
	\item $y = \tg x,\; x \not= \frac{\pi}{2} + \pi k,\ k \in \Z$
\end{enumerate}
\
Обратная функция: $x = f^{-1}(y)$\\
Понятие сложной функции:\\
\includegraphics[width=0.5\linewidth]{images/Функции/Понятие функции/Сложная функция.jpg}\\
g(f(x)) $\subset \Z$
\subsection{Способы задания.}
\begin{enumerate}
	\item Аналитическое задание
	\item Графическое
	\item Неявное задание\\
		f(x, y) = 0\\
		Пример: $x^2 + y^2 = R^2,\; y = \pm\sqrt{R^2 - x^2}$
	\item Параметрическое задание. \\
		\includegraphics[width=0.5\linewidth]{images/Функции/Способы задания/img.jpg}
		\[
			\begin{cases}
				x = x(t)\\
				y = y(t)
			\end{cases}, t \in T
		\]
		\[ 
\begin{cases}
	x = R\cos \phi\\
	y = R\sin \phi
\end{cases}, \phi \in [0, 2\pi)
		\]
\end{enumerate}

\subsubsection{Ограниченность функций.}
\begin{tcolorbox}
	\textbf{Определение.} $f(x)$ ограничена сверху на X, если
	\[ \exists M: \forall x \in X, f(x) < M \]
\end{tcolorbox}
\begin{tcolorbox}
	\textbf{Определение.} $f(x)$ ограничена снизу на X, если
	\[ \exists m: \forall x \in X, f(x) > m \]
\end{tcolorbox}
\begin{tcolorbox}
Если $f(x)$ принимает комплексные значения, то ограниченность $f(x)$ на X
\[ \exists M: \forall x \in X, |f(x)| < M \]
\end{tcolorbox}

\begin{tcolorbox}
	\textbf{Определение.} $f(x): X \in X \to Y$
	\begin{enumerate}
		\item отображение "на", если
			\[ \forall y \in Y\; \exists x \in X: g = f(x) \]
		\item отображение "в", если
			\[ \forall x \in X\; \exists y \in Y: y = f(x) \]
		\item "взаимно однозначное отображение"
			\[ \forall x \in X\; \exists! y \in Y: y = f(x) \]
			\[ \forall y \in Y\; \exists! x \in X: y = f(x) \]
	\end{enumerate}
\end{tcolorbox}

\begin{tcolorbox}
	\textbf{Определение.} $f(x): X \to Y$
	\[ \sup_X f(x)=\sup f(X) \]
\end{tcolorbox}
\[ M = \sup_X f(x) \]
если 
\begin{enumerate}
	\item $\forall x \in X\; f(x) \leq M $
	\item $\forall M` < M\; \exists x \in X: f(x) > M` $
\end{enumerate}

\begin{tcolorbox}
	\textbf{Определение.} Если $f(x)$ не ограничена сверху на X, то $\exists \{ x_n \} \supset X $
	\[ f(x_n) \xrightarrow[n\to\infty]{} +\infty \]
\end{tcolorbox}

\begin{tcolorbox}
	\textbf{Определение.} $f(x)$ принимает max в $x_0 \in X$, если
	\[ \forall x \in X\;\; f(x) \leq f(x_0) \]
	\[ f(x_0) = \max_X f(x) \]
\end{tcolorbox}

\[ f(x) = x in [0, 1) \]
\[ \min_{[0,1)} f(x) - \inf){[0, 1)} f(x) = 0\]
\[ \sup_{[0,1)} f(x) =1\;\; \max_{[0,1)} f(x) \not \exists \]

\subsection{Предел функции в точке.}

На протяжении всего параграфа: $f(x)$ на $\mathring U_a = U_a / \{a\}$
\[ a \in (d, c) = U_a \]
\includegraphics[width=0.5\linewidth]{images/Функции/Предел функции в точке/График.jpg}
\begin{tcolorbox}
	\textbf{Определение. Предел по Коши. } $f(x) \to A$ при $x \to a$, если 
	\[ \forall \eps > 0\; \exists \delta_\eps > 0: \forall x: 0<|x-a| < \delta_\eps\;\; |f(x) - A| < \eps \]
\end{tcolorbox}

\begin{tcolorbox}
	\textbf{Определение. Существование предела функции в точке.} $f(x)$ имеет предел при $x \to a$, если 
	\[ \exists A\;\; \forall \eps > 0\; \exists \delta_\eps > 0: \forall x: 0<|x-a| < \delta_\eps\;\; |f(x) - A| < \eps  \]
\end{tcolorbox}

\begin{tcolorbox}
	\textbf{Определение. Предел по Гейне. } $f(x)$ имеет придел при $x \to a$, если\\
	$\forall \{x_n\}\; \forall n, x_n \in \mathring U_a, x_n \to a\; \{f(x_n)\}$ сходится
\end{tcolorbox}
Не обязательно, чтобы предел функции и последовательности был одним и тем же.

\begin{tcolorbox}
	\textbf{Теорема. Предел в смысе Коши.}
	$\exists \Leftrightarrow \exists$ предел по Гейне.
\end{tcolorbox}
\begin{tcolorbox}[title=Доказательство]
	$\Rightarrow \exists A: \forall \eps > 0\; \exists \delta_\eps > 0$\\
	\[ \forall x \in (a - \delta, a + \delta) / \{a\} \;\; |f(x) - A| < \eps \]
	Пусть $\{ x_n \}: \forall n\; x_n \in \mathring U_a, x_n \xrightarrow[n \to \infty]{} a$
\[ \exists N_{delta_\eps} \in \N: \forall n > N_{\delta_\eps}\;\; 0 <|x_n - a| < \delta_\eps \]
\[ \Rightarrow x_n \in (a - \delta_\eps, a + \delta_\eps) / \{a\} \Rightarrow |f(x) - A| < \eps \]

	$\Leftarrow$ Пусть $f(x)$ имеет в a предел в смысле Гейне.
	\begin{enumerate}
	\item $\lim f(x_n)$ не зависит от $\{x_n\}$\\
		Пусть не так зависит), тогда $\exists \{x_n`\}$\\
		\[ x_n` \to a, x_n` \not= a \; \forall n: f(x_n`) \to A` \]
		\[ \exists \{ x_n ``\}, x_n` \to a, x_n`` \not = a\;\; \forall n, f(x_n``) \to A`` \]
		\includegraphics[width=0.5\linewidth]{images/Функции/Предел функции в точке/Гейне.jpg}
		\[ x_{2k} = x_k`` \]
		\[ x_{2k-1} = x_k` \]
		\[ \forall n\;\; x_n \to a, x_n \not = a \]
		По определению предела по Гейне:
		\[ \exists A: f(x_n) \to A: f(x_{2n}) - to A``, f(x_{2n-1}) - to A` \]
		таким образо: $ \exists A: \forall \{ x_n \} $
		\[ f(x_n) \to A \]
	\item $ f(x) \xrightarrow[]{x \to a} A $ по Коши\\
		От противного:\\
		Пусть $ f(x) \not \xrightarrow[]{a \to a} A$ по Коши
		\[ \exists \eps > 0: \forall \delta > 0\; \exists x_n: 0 < |x-a| < \delta\;\; |f(x) - A| \geq \eps \]
		\[ x_1: 0 < |x_1 - a| < 1/1\;\; |f(x_1) - A| \geq \eps \]
		\[ x_2: 0 < |x_2 - a| < \frac{1}{2}\;\; |f(x_2) - A| \geq \eps \]
		\[...\]
		\[ x_n: 0 < |x_n - a| < \frac{1}{n}\;\; |f(x_n) - A| \geq \eps \]
		\[...\]
		$ \{x_n\} $, $x_n \to a, x_n \not = a, \forall n \Rightarrow f(x_n) = A$ противоречие

	\end{enumerate}
\end{tcolorbox}

Пример эквиволентности определений по Коши и по Гейне: $y = \sin \frac{1}{x}$\\
\begin{center}
	\includegraphics[width=0.5\linewidth]{images/Функции/Предел функции в точке/Пример.jpg}\\
\end{center}
Вопрос на доп баллы: Приевести пример функции непрерывной на интервале, которая не имела бы придела при x стремящемся к концу интревала.\\
$\forall x_n \to 0 \; \{f(x_n)\} $ - сходится
\[ \frac{1}{x_n} = \frac{\pi}{2} + \pi n, n \in \N  \]
\[ x_n = \frac{1}{\frac{\pi}{2} + \pi n}\;\;\; f(x_n) = \sin (\frac{\pi}{2} + \pi n) = (-1)^n \]
\[ x_n` = \frac{1}{\frac{\pi}{2} + 2\pi n}\;\; f(x_n) = 1 \]
\[ x_n`` = \frac{1}{\frac{3\pi}{2} + 2\pi n}\;\; f(x_n) = -1 \]

\subsection{Односторонние пределы.}


\begin{tcolorbox}
	\textbf{Определение. }Предел справа.\\
	$f(x)$ на $(a, b) \to \R$\\
	$ f(x) \to A$, при $x \to a + 0$\\
	$ f(x) \to A$, при $x \to a, x > a$ \\
	$f(x) \to A$, при $x \to a$ справа $\Leftrightarrow \forall \eps > 0 \exists \delta > 0 \forall x \in (a, a+ \delta) (\forall x: |x-a| < delta, x > a) |f(x) - A| < \eps$
\end{tcolorbox}

\begin{tcolorbox}
	Опеделение. Предел слева.
	\[ B = \lim_{x \to b-0} f(x) \Leftrightarrow \forall \eps > 0\; \forall x\; b - \delta < x < b,\; |f(x) - B| < \eps \]
\end{tcolorbox}

\begin{tcolorbox}
	Определение по Гейне. 
	\[ \exists \lim_{x \to a + 0} f(x) \Leftrightarrow \forall \{x_n\} \supset (a, b)\; x_n \xrightarrow[]{n\to \infty} a \;\; {f(x_n)} - converges \]
\end{tcolorbox}

\textbf{Утверждение.} Существование предела по Коши справа или слева равносильно существованию соответствующего одностороннего предела по Гейне.

\textbf{Теорема.} $\exists \lim_{x \to a} f(x) \Leftrightarrow$ 
\[
\begin{cases}
	\exists \lim_{x \to a + 0} f(x)\\
	\exists \lim_{x \to a - 0} f(x)\\
	\lim_{x \to a + 0} f(x) = \lim_{x \to a - 0} f(x)\\
\end{cases} 
\]
\begin{tcolorbox}[title=Доказательство]
	$\Rightarrow:\; A = \lim_{x \to a} f(x)$ По определению $\forall \eps > 0\; \exists \delta_\eps > 0: \forall x\; (0 < |x - a| < \delta_\eps \Leftrightarrow x \in (a - \delta_\eps, a) \cup (a, a + \delta_\eps)$
	\[ x \in (a - \delta_\eps, a) \Rightarrow A = \lim_{x \ to a -0} f(x) \]
	\[ x \in (a, a + \delta_\eps) \Rightarrow A = \lim_{x \ to a + 0} f(x) \]

	$\Leftrightarrow:$ 
	\[ \lim_{x \to a + 0} f(x) = \lim_{x \to a - 0} f(x) := A\]
	\[ A=  \lim_{x + 0} f(x) \Rightarrow \forall \eps > 0 \exists \delta_\eps > 0\; \forall x \in (a, a + \delta_\eps) \;\; |f(x) - A| < \eps \]
	\[ A = \lim_{x - 0} f(x) \Rightarrow \forall \eps > 0 \exists \delta_\eps` > 0\; \forall x \in (a - \delta_\eps`, a) \;\; |f(x) - A| < \eps \]

	\[ \delta_\eps = \min{\delta_eps, \delta_\eps`} \;\; \forall x: 0 < |x - a| < \delta_\eps\;\; |f(x) - A| < \eps \]
\end{tcolorbox}

\subsection{Критерий Коши. Существование предела функции в точке.}
\textbf{Теорема. Критерий Коши.} $ f(x) $ на $\mathring U_a$
\[ \exists \lim_{x \to a} f(x) \Leftrightarrow \forall \eps > 0\; \exists \delta_\eps > 0: \forall x`, x`` \in (a - \delta_\eps, a + \delta_\eps)/\{a\}\;\; |f(x`) - f(x``)| < \eps \]
\begin{tcolorbox}[title=Доказательство]
	$\Rightarrow: \exists \lim_{x \to a} f(x) := A$
	\[ \forall \eps > 0\; \exists \delta_\eps\; \forall x: 0< |x - a| < \delta_\eps\;\; |f(x) - A| < \frac{\eps}{2} \]
	\[ \forall x`, x``: 0 < |x` - a| < \delta_\eps, 0 < |x`` - a| < \delta_\eps \]
	\[ |f(x`) - f(x``)| = |(f(x`) - A) - (f(x``) - A)| \leq |f(x`) - A| + |f(x``) - A| < \frac{\eps}{2} + \frac{\eps}{2} = \eps\]
	$\Leftarrow: $ Пусть $\forall \eps > 0\; \exists \delta_\eps > 0: \forall x`, x`` \;\; 0 < |x` - a| < \delta_\eps\;\; 0 < |x`` - a| < \delta_\eps \;\; |f(x`) - f(x``)| < \eps$\\
	\[ \forall \{ x_n \} \subset \mathring U_a\;\; x_n \to a \Rightarrow \exists n_{\delta_\eps} \]
	\[ 0 < |x_n - a| < \delta_\eps\;\; \forall n > n_{\delta_\eps} \Rightarrow \forall n, m > n_{\delta_\eps} \;\; |f(x_n) - f(x_m)| < \eps \]
	Что значит, что последовательсноть $\{ f(x_n) \}$ - фундаментальная, значит по Критерию Коши для последовательностей $\{f(x_n)\}$ - сходится. $\Rightarrow \exists \lim_{a\to a} f(x)$ в смысле Гейне.
\end{tcolorbox}

\textbf{Отрицание.} $f(x)$ определена на $ \mathring U_a$
\[ \not \exists \lim_{x \to a} f(x) \Leftrightarrow \exists \exp > 0\; \forall \delta\; \exists x`, x``\; 0 < |x` - a| < \delta \;\; 0 < |x`` - a| < \delta |f(x`) - f(x``)| \geq \eps \]
\textbf{Пример. } 
\begin{enumerate}
	\item $f(x) = \frac{1}{x}$ на (0, 1)
	Нет предела при $x \to 0 + 0$\\
	Пусть $\eps = 1\; \forall \delta\; \forall N > \frac{1}{\delta}$
	\[ x`:= \frac{1}{N}\;\;\; x`` := \frac{1}{N+1} \]
	\[ |f(x`) - f(x``)| = |\frac{1}{\frac{1}{N}} - \frac{1}{\frac{1}{N+1}}| \geq \eps \] 
	\item $f(x) = \sin \frac{1}{x}$ на (0, 1)\\
		$\eps := 1$ Сами. 

\end{enumerate}

\subsection{Пределы при стремлении к бесконечности. Бесконечные пределы.}
\textbf{Определение.} $\eps$ окрестность $\infty\ (-\infty; - \eps) \cup (\eps; +\infty)$\\
\textbf{Определение.} $\forall a < b$ окрестность $\infty\; (-\infty; a) \cup (b; +\infty)$\\
\textbf{Определение.} $\forall b$ окрестность $+\infty\; (b; +\infty)$\\
\textbf{Определение.} $\forall a$ окрестность $-\infty\; (-\infty; a)$\\
\[ \forall a \in \R \rightarrow a + 0, a-0 \]
\textbf{Определние.} $f(x)$ определена на $\mathring U_a \rightarrow \R$ \\
\begin{enumerate}
	\item $ f(x) \to \infty, x \to a \Leftrightarrow \forall \eps > 0\; \exists \delta_\eps > 0: \forall x\; 0< |x - a| < \delta_\eps \;\; |f(x)| > \eps \Leftrightarrow f(x) \in U_{\infty, \eps}$
	\item $f(x) \to +\infty$ при $x \to a, \forall \eps > 0 \;\; \exists \delta_\eps > 0: \forall x\; 0 < |x - a| < \delta_\eps \;\; f(x) > \eps$
	\item $f(x) \to -\infty$ при $x \to a \; \forall \eps > 0 \exists \delta_\eps > 0: \; 0 < |x - a| < \delta_\eps \;\; f(x) < -\eps $
\end{enumerate}
Extra: $f(x)$ на $U_{a + 0}: f(x) \to - \infty, x \to a + 0$
\[ \forall \eps > 0 \exists \delta_\eps > 0 \; \forall x \in (a, a + \delta_\eps) \]
\[ f(x) \in (-\infty; -\eps ) \]
\[ f(x) < -\eps \]

\subsection{Свойства функций, имеющих предел}
\textbf{Теорема 1.} Если предел функции в точке существует, то он единственен. \\
Ну он опять свой доказательство придумал. 
\begin{tcolorbox}[title=Доказательство от противного]
	Пусть $f(x) \to A` , x \to a$\\
	$f(x) \to A``$, при $x \to a$ и $A` \not= A``$\\
	$\eps := \frac{|A` - A``|}{4}$\\
	$\exists \delta` > 0: \forall x 0 < |x - a| < \delta`\; |f(x) - A`| < \eps$\\
	$\exists \delta`` > 0: \forall x 0 < |x - a| < \delta``\; |f(x) - A``| < \eps$\\
	$ \delta := \min\{ \delta`, \delta`` \} > 0 $\\
	$ \forall x: 0 < |x - a| < \delta $\\
	$ |f(x) - A``| = (f(x) - A`) - (A`` - A`)| \geq |A` - A``| - |f(x) - A`| > \frac{3}{4} |A` - A``|$ \\
	$|f(x) - A`| < \frac{|A` - A``|}{4}$\\
\end{tcolorbox}
Удостоверится, что a и A мжет быть $\infty, \pm \infty$.\\
\textbf{Теорема 2.} $f(x) \to A$ при $x \to a \Rightarrow |f(x)| \to |A|$, при $x \to a$. 
\begin{tcolorbox}[title=Доказательство]
	\[ \forall \eps > 0\; \exists \delta > 0: \forall x\; 0 < |x - a| < \delta\; |f(x) - A| < \eps \]
	\[ ||f(x)| - |A|| \leq |f(x) - A| < \eps \]
\end{tcolorbox}
\textbf{Теорема 3. О сохранением знака, только $\R$.} \\
Если $f(x) \to A \not = 0, x \to a$, то $\exists \delta > 0\; \forall x\; 0 < |x - a| < \delta \;\; \sign f(x) = \sign A$
\begin{tcolorbox}[title=Доказательство. Непосредственная проверка.]
	\[ \eps := \frac{|A|}{2}\; \exists \delta > 0: \forall 0 < |x - a| < \delta\; |f(x) - A| < \eps \]
	\[ A - \eps < f(x) < A + \eps \]
	\begin{enumerate}
		\item $A > 0$:
			\[ f(x) > A - \eps = A - \frac{A}{2} = \frac{A}{2} > 0 \]
		\item $ A < 0 $:
			\[ f(x) < A + \eps = A - \frac{A}{2} = \frac{A}{2} < 0 \]
	\end{enumerate}
\end{tcolorbox}
\textbf{Теорема 4.} Если $f(x) \xrightarrow[x\to a]{} A$ число $\Rightarrow \exists \mathring U_a$ в которой $f(x)$ ограничена $(f(x) \in B(\mathring U_a))$.
\begin{tcolorbox}[title=Доказательство]
	\[ \eps := 30 \exists \delta > 0: \forall x\; 0 < |a - a| < \delta\; |f(x) - A| < 30 \]
	\[ |f(x)| = |(f(x) - A) + A| \leq |f(x) - A| + |A| < |A| + 30 \]
	Таким образом функция ограничена. 
\end{tcolorbox}
\textbf{Теорема 5. Арифмитические свойства.}\\
Пусть $\exists \lim_{x\to a} f(x), \lim_{x\to a} g(x)$. Тогда \\
\begin{enumerate}
	\item $\exists \lim_{x\to a} (f(x) \pm g(x))$ и $\lim_{x\to a} (f(x) \pm g(x)) = \lim_{x\to a} f(x) \pm \lim_{x \to a} g(x)$
	\item $\exists \lim_{x\to a} (f(x) \cdot g(x))$ и $\lim_{x\to a} (f(x) \cdot g(x)) = \lim_{x\to a} f(x) \cdot \lim_{x\to a} g(x)$
	\item Если $\lim_{g(x)} \not= 0$ и $g(x)\not=0$, то $\exists \lim_{x\to a} \frac{f(x)}{g(x)}$ и $\lim_{x\to a} \frac{f(x)}{g(x)} =  \frac{\lim_{x\to a} f(x)}{\lim_{x\to a} g(x)}$
\end{enumerate}
\begin{tcolorbox}
	Эквивалентность существования пределов по Гейне и Коши и соответствующие свойства для последовательностей.\\
	Сами дакажите. @Теляк. 
\end{tcolorbox}
\textbf{Теорема 6. Предельный переход в неравенствах.}\\ 
$f(x) \leq g(x)$ при $x, a\in (c, d), x\not= a$\\
Если $\exists \lim_{x\to a} f(x), \exists \lim_{x\to a}g(x) \Rightarrow \lim_{x\to a} f(x) \leq \lim_{x\to a} g(x)$
\begin{tcolorbox}[title=Доказательство. Эквивалентность пределов по Гейне и по Коши и теорема для последовательностей. ]
	
	
\end{tcolorbox}
\textbf{Теорема 7. О зажатой функции.}\\
\[ \forall a, x \in (c, d), x \not= a \;\; f(x) \leq \phi(x) \leq g(x)\; \&\; \lim_{x\to a} f(x) = \lim_{x\to a} g(x) \Rightarrow \exists \lim_{x\to a} \phi(x)\; \&\; \lim_{x\to a} \phi(x) = \lim_{x\to a} g(x) \]
\begin{tcolorbox}[title=Доказательство.]
	\[ A:= \lim_{x\to a} f(x) \]
	\[ \forall \eps > 0 \delta_f > 0: \forall 0 < |x - a|<\delta_f\; |f(x) - A| < \eps \Leftrightarrow A - \eps \leq f(x) \leq A + \eps \]
	\[ \exists \delta_g > 0: \forall x\; 0 < |x - a| < \delta_g \]
	\[ |g(x) - A| < \eps \Leftrightarrow A - \eps < g(x) < A + \eps \]
	\[ \delta := \min\{\delta_f, \delta_g \} > 0 \]
	\[ \forall x: 0 < |x - a| < \delta  \]
	\[ A - \eps < f(x) \leq \phi(x) \leq g(x) < A + \eps \]
	\[A - \eps < \phi(x) < A + \eps \]
	\[ \phi(x) \xrightarrow[x\to a]{} A \]
\end{tcolorbox}
\textbf{Теорема 8.} $f(x) \xrightarrow[x\to a]{} A \Leftrightarrow f(x) = A \ alpha(x)\;\; (\alpha(x) \xrightarrow[x\to a]{} 0)$\\

Продумайте какие Т про арифм действия переносятся на предел к бесконечности.
\begin{tcolorbox}[title=Доказательство]
$\Rightarrow$ По определению $\forall \eps > 0\; \exists \delta > 0\; |f(x) - A| < \eps, \alpha(x) := f(x), \forall x\; 0 < |x - a| < \delta$\\
$\alpha(x) \to 0$ по определению $f(x) = A + \alpha(x)$\\
$\Leftarrow$ Пусть $f(x) = A + \alpha(x)$, где $\alpha(x)$ - б.м., при $x \to a$\\
По определению $\forall \eps > 0\; \exists \delta > 0: \forall x\; 0< |x - a| < \delta\; |\alpha(x)| < \eps$\\
$|\alpha(x)| = |f(x) - A| \Rightarrow f(x) \to A, x \to a$
\end{tcolorbox}

\textbf{Теорема 9.}
\begin{enumerate}
	\item Если $f(x) \xrightarrow[x\to a]{}$, то $\frac{1}{f(x)} \to 0$
	\item $\alpha(x) \xrightarrow[x\to a]{} 0, \alpha(x) \not= 0 \Rightarrow \frac{1}{\alpha(x) \xrightarrow[x \to a]{} \infty}$
\end{enumerate}
\begin{tcolorbox}[title=Доказательство]
	\begin{enumerate}
		\item $\forall \eps > 0\; \exists \delta_\eps > 0: \forall x\; 0 < |x - a| < \delta_\eps\; |f(x)| > \frac{1}{\eps} \Leftrightarrow |\frac{1}{f(x)}| < \eps$
		\item $\forall \eps > 0\; \exists \delta_\eps > 0: \forall x\; 0 < |x - a| < \delta_\eps |\alpha(x)| < \frac{1}{\eps} \Leftrightarrow |\frac{1}{\alpha(x)}| > \eps$
	\end{enumerate}
\end{tcolorbox}

\subsection{Некоторое важные пределы.}
Если понять какие моменты обман, то на экзамене будет + в карму. \\
\textbf{Теорема 1.} $(1 + \frac{1}{x})^x \xrightarrow[x\to \infty]{} e$
\begin{tcolorbox}[title=Доказательство]
	\begin{enumerate}
		\item $x \to +\infty$
			\[ \forall x > 1 \; 0 < [x] \leq x <  [x] + 1 \]
		\[  \frac{1}{[x]} \geq \frac{1}{x} > \frac{1}{[x] + 1} \]
		\[ 1 + \frac{1}{[x]} \geq 1 + \frac{1}{x} > 1 + \frac{1}{[x] + 1} \]
		\[ (\frac{1}{[x]})^{[x]+1} > (\frac{1}{x})^x > (\frac{1}{[x] + 1})^{[x]} \]
		Вспомним последовательсноти: $(1 + \frac{1}{n})^{n+1} = (1 + \frac{1}{n})^n (1 + \frac{1}{n}) \xrightarrow[n\to\infty]{} e $
		\[ (1 + \frac{1}{n+1})^n = (1 + \frac{1}{n+1})^{n+1} / (1 + \frac{1}{n+1}) \to e \]
	\end{enumerate}
\end{tcolorbox}

\[ x\to\infty\;\; \forall x > 1 \]
\[ (\frac{1}{[x]})^{[x]+1} > (\frac{1}{x})^x > (\frac{1}{[x] + 1})^{[x]} \]
\[ (1 + \frac{1}{n+1})^n \xrightarrow[x\to\infty]{} e \;\;\; (1 + \frac{1}{n})^{n+1} \xrightarrow[x\to\infty]{} e \]
\[ \forall \eps > 0 \;\; \exists n_\eps` : \forall n > n_\eps` \]
\[ e - \eps < (1 + \frac{1}{n})^n < e + \eps \]
\[ \exists n_\eps`` : \forall n > n_\eps`` \]
\[ e - \eps < (1 + \frac{1}{n})^{n+1} < e + \eps \]
\[ n_\eps := \max\{ n_\eps`, n_\eps`` \} \]
\[ \forall n > n_\eps \]
\[ \forall x > n_\eps + 1 \]
\[ e - \eps < (1 + \frac{1}{n})^n < (1 + \frac{1}{x})^x < (1 + \frac{1}{n})^{n+1} < e + \eps \]

\begin{enumerate}
	\item[б) ]  $x\to\-\infty$\\
		\[ t = -x\;\; t \to +\infty \]
		\[ (1 + \frac{1}{[x]})^{x} = (1 - \frac{1}{x})^{-t} = (\frac{t}{t - 1})^{t}  = (1 + \frac{1}{t-1})^{t-1} (1 + \frac{1}{t  1}) \xrightarrow[n\to \infty]{} e \]
	\item[в) ] $x\to \infty$\\
		\[ \forall \eps > 0\;\; \exists \delta_\eps^+ > 0 \;\; \forall x > \delta_\eps^+ \]
		\[ \exists \delta_\eps^- > 0:\; \forall x < -\delta_\eps^- \]
		\[ |(1 + \frac{1}{x})^x - e| < \eps \]
		\[ \delta_\eps := \max\{ \delta_\eps^+; \delta_\eps^- \} (>0) \]
		\[ \forall x: |x| > \delta_\eps \]
\end{enumerate}

\textbf{Следствие.} $(1 + \frac{1}{x})^{\frac{1}{x}} \xrightarrow[x\to 0]{} e$
\begin{tcolorbox}[title=Доказательство]
	\[ t = \frac{1}{x} \to \infty,\; x\to0 \]
	\[ x = \frac{1}{t} \;\;\; (1 + x)^{\frac{1}{x}} = (1 + \frac{1}{t})^t \to e \]
\end{tcolorbox}

\textbf{Теорема 2.} $ \frac{\sin x}{x} \xrightarrow[x\to0]{} 1 $
\begin{center}
	\includegraphics[width=0.5\linewidth]{images/Функции/Особые пределы/sin}
\end{center}

\[ S = \displaystyle\frac{1}{2} r^2\alpha = \frac{1}{2}r(r \cdot \alpha) = \frac{1}{2}rl \]
\begin{center}
	\includegraphics[width=0.5\linewidth]{images/Функции/Особые пределы/sin}
\end{center}
\[ S_{OCD} = \frac{1}{2}(\cos^2 x) < \]
\[ S_{OAB} = \frac{1}{2}\cdot 1\cdot \sin x < \]
\[ S_{sector\;OAB} = \frac{1}{2} \cdot 1^2 \cdot x \]
\[ x\cos^2 x < \sin x < x \]
\[ \forall x, |x| < \frac{\pi}{2}, x\not=0\;\;\;\;\;\;\;x \cos^2 x <\frac{\sin x}{x} < 1 \]
\[ \forall \R \;\; x \in  0\leq |\sin x| \leq |x| \]
\[ x \to 0\;\; |x| \to 0 \Rightarrow |\sin x| \to 0 \]
\[ \forall x,\; |x| < \frac{\pi}{2} \]
\[ 1 - \sin^2 x = \cos^2 x< \frac{\sin x}{x} < 1,\;\; x \to 0 \]
\[ \Rightarrow \frac{\sin x}{x} \to 1 \]

\subsection{Монотонные функции.}
$ f(x)$ на $<a,b>$ \\
\textbf{Определение.} $f(x)\uparrow$ на $<a,b>$, если $\forall x`, x`` \in <a,b> x` < x`` \Rightarrow f(x`) \leq f(x``)$ (монотонно возрастает нестрого)\\
\textbf{Определение.} $f(x)\uparrow$ на $<a,b>$, если $\forall x`, x`` \in <a,b> x` < x`` \Rightarrow f(x`) < f(x``)$ (монотонно возрастает строго)\\
Убываение - анал.\\

\textbf{Теорема.} $f(x) \uparrow$ на $(a,b) \Rightarrow $
\begin{enumerate}
	\item Если F(x) ограничена снизу на $(a, b)$
		\[ \exists \lim_{x \to a+0} f(x) = \inf_{(a,b)} f(x) \]
	\item Если F(x) ограничена сверху на $(a, b)$
		\[ \exists \lim_{x\to b-0} f(x) = \sup_{(a,b)} f(x) \]
	\item Если F(x) не ограничена сверху на $(a, b)$
		\[ f(x) \xrightarrow[x \to b - 0]{} +\infty \]
	\item Если F(x) не ограничена снизу на $(a, b)$
		\[ f(x) \xrightarrow[x \to a + 0]{} -\infty \]
\end{enumerate}
\begin{tcolorbox}[title=Доказательство]
	При $x\to a + 0$
	\begin{enumerate}
		\item $f(x)$ ограничена снизу на $(a, b)$
			\[ \Rightarrow \exists \inf_{(a,b)} f(x)=:A\in\R \]
			- это Т Вейерштрасса
			\[ \forall \eps > 0  \]
			\begin{center}
				\includegraphics[width=0.5\linewidth]{images/Функции/Монотонность/Док-во ограниченности}
			\end{center}
			$A + \eps$ - не нижняя грань $f(x)$ на $(a,b)$, т.е. 
			\[ \exists x_\eps \in (a, b) : A\leq f(x\eps) < A + \eps \Rightarrow x \in (a, x_\eps) \]
			\[ A \leq f(x) \leq f(x_\eps) < A + \eps \Rightarrow \]
			\[ |f(x) - A| < \eps \]
			\[ \delta_\eps := x_\eps - a > 0 \]
			\[ \forall x: 0 < x - a < \delta_\eps \]
		\item[2.] $f(x)$ не ограничена снизу на (a,b)
			\[ \forall \eps > 0 \exists x_\eps \in (a, b): f(x_\eps) < -\eps \]
			\begin{center}
				\includegraphics[width=0.5\linewidth]{images/Функции/Монотонность/Док-во ограниченности3}
			\end{center}
			\[ \forall x \in (a, x_\eps) \]
			\[ f(x) \leq f(x_\eps) < -\eps - \eps \]
			\[ \delta_\eps := x_\eps-a > 0 \]
			\[ \forall x \in (a, a+\delta_\eps) \]
			т.е. $f(x) \xrightarrow[x \to a + 0]{} \infty$\\
			\textbf{Следствие 1.} Для $f(x) \downarrow$ на $(a, b)$\\
			\textbf{Следствие 2.} $f(x)$ монотонна на $(a, b)$
			\[ \forall x_0 \in (a, b) \;\; \exists \lim_{x\to x_0 + 0} f(x) \;\; \exists \lim_{x\to x_0 - 0} f(x) \]
	\end{enumerate}
\end{tcolorbox}

\subsection{Непрерывность в точке}
\textbf{Определение.} $f(x)$ на $U_a \;\; f(x)$ непрерывна, если $\lim_{x\to a} f(x) = f(a)$ (по Коши) \\
Определение непрерывности по Гейне: Он должен ответить на этот вопрос на след лекции!!
\textbf{Определение.} $f(x)$ на $\mathring U_a$ разрывна в c, если 
\begin{enumerate}
	\item $f(x)$ не определена в (.) a
	\item $f(x)$ не непрерывна в (.) a
\end{enumerate}
\textbf{Определение.} a - точка разрыва
\begin{enumerate}
	\item Если $\exists \lim _{x\to a+ 0} f(x) \& \exists \lim _{x\to a-0} f(x)$, то a - точка разрыва I рода. 
		\begin{enumerate}
			\item Если $\lim _{x\to a+ 0} f(x) = \lim _{x\to a+ 0} f(x)$, то у $f(x)$ в (.) c - устранимый разрыв. 
			\item Иначе - неустранимый
		\end{enumerate}
	\item Если хотя-бы 1 из пределов не существует, то разрыв II рода.
\end{enumerate}

\textbf{Примеры.}
\begin{enumerate}
	\item $\sign x = (sgn\ x) = \begin{cases}
		1, x > 0\\
		0,x=0\\
		-1,x<0
	\end{cases}$\\
	x = 0 $\Rightarrow$ разрыв I рода. 
	\item $\sign ^2 x \;\; x = 0$ - устранимый разрыв
	\item $y = \sin \frac{1}{x}$, $x\not= 0 $\\
		$x = 0$ - ограниченный разрыв 2 рода. Предела здесь нет. 
	\item $y = \frac{1}{x} \sin \frac{1}{x}$, $x = 0$ - не ограничена разрыв 2 рода
	\item $f(x) \ \begin{cases}
		\frac{1}{x}, x\not=0\\
		0,x=0 
	\end{cases},\;\; x=0$ - бесконечный разрыв 2 рода
	\item Если f(x) монотонна на <a,b> то у f(x) могут быть только разрывы 1 рода. 
\end{enumerate}
\textbf{Определение.} f(x) непрерывна в $x_0$ справа, если $\exists \lim_{x \to x_0 + 0} f(x)$ и $\lim_{x \to x_0  + 0} f(x) = f(x_0)$\\
\textbf{Определение.} f(x) непрерывна в $x_0$ слева, если $\exists \lim_{x \to x_0 - 0} f(x)$ и $\lim_{x \to x_0  - 0} f(x) = f(x_0)$\\
\textbf{Пример.} $f(x) = [x], x \in \R$
Непрерывна справа на $\R$ \\
\textbf{Теорема 1.} $f, g$ неопределены в $x_0 \Rightarrow f \pm g, f \cdot g, \frac{f}{g}$ (если $f(x_0) \not=0$) неопределена в $x_0$. \\
\textbf{Теорема 2.} f непрерывна в $f(x_0) \not= 0$, то f(x) сохраняет знак в некоторой окрестности $U_{r_0}$\\
\textbf{Теорема 3.} $y = f(x)$ неопределена в $x_0$, $x = \phi(t)$ непрерывна в $t_0, \phi(t_0) = x_0 \Rightarrow f(\phi(t))$ - непрерывна в $t_0$
\begin{tcolorbox}[title=Доказательство]
	\[ \forall \eps > 0\; \exists \eta_\eps > 0: \forall x \;\; |x - x_0| < \eta_\eps  \Rightarrow |f(x) - f(x_0)| < \eps  \]
	\[ \exists \delta_{\eta_\eps} (=\delta_\eps): \forall t \;\; |t - t_0| < \delta_\eps \;\; |\phi(t) - \phi(t_0)| < \eta_\eps \]
	\[ \forall t |t - t_0| < \delta_\eps \]
	\[ |f(\phi(t)) - f(\phi(t_0)) | = |f(x_t) - f(x_0)| < \eps \]
	По выбору $\delta_\eps \;\; |x_t -  x_0| < \eta_\eps $
\end{tcolorbox}

\subsection{Бесконечно малые и бесконечно больше функции}

\textbf{Определение.} $\alpha(x)$ на $\mathring U_a$ - б.м. при $x \to a$, если $\alpha_x \xrightarrow[x\to a]{} 0$
\[ \forall \eps > 0 \;\; \exists \delta_\eps > 0: \forall x\; 0 < |x - a| < \delta_\eps \;\; |\alpha(x) < \eps| \]
\textbf{Теорема 1.} 
\begin{enumerate}
	\item $\alpha(x), \beta(x)$ - б.м., при $x\to a \Rightarrow \alpha(x) \pm \beta(x)$ б.м. при $x\to a$
	\item $\alpha(x) \xrightarrow[x\to a]{} 0, f(x) \in B(U_a)$ 
		$\Rightarrow \alpha(x) f(x)$ - б.м., при $x \to a$
\end{enumerate}
\begin{tcolorbox}[title=Доказательство]
	\begin{enumerate}
	\item \[ \forall \eps > 0\; \exists \delta_\eps` > 0\;\; \forall x \;\; 0 < |x - a| < \delta_\eps` \;\; |\alpha(x)|< \frac{\eps}{2} \] 
	\[ \exists \delta_\eps`` > 0\;\; \forall x \;\; 0 < |x - a| < \delta_\eps`` \;\; |\alpha(x)|< \frac{\eps}{2} \]
	\[ \delta_\eps = \min\{ \delta_\eps`,\delta_\eps`` \} > 0 \;\; \forall x \;\; 0 < |x - a| < \delta_\eps \]
	\[ |\alpha(x) \pm \beta(x)| \leq |\alpha(x)| + |\beta(x)| < \frac{\eps}{2} + \frac{\eps}{2} = \eps \]
	\item анал
	\end{enumerate}
\end{tcolorbox}

\textbf{Теорема 2.} 
\begin{enumerate}
	\item  $f(x) \xrightarrow[x\to a]{} A \Leftrightarrow f(x) = A + \alpha(x) $\\
\textbf{Определение.} $f:\mathring U_a, f(x)$ - б.б. при $x\to a$, 
\[ \forall \eps > 0 \;\; \exists \delta_\eps > 0: \forall x \;\; 0 < |x - a| < \delta_\eps \;\; |f(x)| > \eps,\; f(x) \xrightarrow[x\to a]{} \infty \]
	\item $f(x) \to +\infty,\; \forall \eps > 0\; \exists \delta_\eps > 0: \forall x \; 0 < |x - a| < \delta_\eps \;\; f(x) > \eps$
	\item $f(x) \to -\infty \;\; |f(x)| < -\eps$
\end{enumerate}
\textbf{Теорема 3.} 
\begin{enumerate}
	\item $f(x) \xrightarrow[x\to a]{} \infty  \Leftrightarrow \frac{1}{f(x)} \xrightarrow[x\to a]{} 0$
	\item $\exists \mathring U_a: \alpha(x) \not=0, x\in \mathring U_a$\\
	\[ \alpha(x) \xrightarrow[x\to a]{} 0 \Leftrightarrow \frac{1}{\alpha(x)} \xrightarrow[x\to a]{} \infty \]
\end{enumerate}

\subsection{Непрерывность на промежутке}
\textbf{Определение.} f(x) непрерывность на <a,b>, если f(x) на непрерывность в $\forall x \in <a, b>$ (относительно <a,b>), $f(x) \in C<a,b>$\\
C - непрервсноть на отрезке. \\
B - ограниченная\\
Пример \\
\begin{center}
	\includegraphics[width=0.5\linewidth]{images/Функции/Непрерывность/Непрерывность на промежутке}
\end{center}

\textbf{Теорема 1. Первая теорема Вейерштрасса} $f(x) \in C[a,b] \Rightarrow f(x) \in B[a, b] $ 
\begin{tcolorbox}[title=Доказательство от противного]
	Предположим, что $f(x) \not\in B([a,b])$, т.е. $\forall M \; \exists x_M \in [a, b] \; |f(x)| > M$
	\[ \{ x_n \}: \forall n \in \N \; x_n \in [a,b] |f(x_n)| > n \]
	\[ \{ x_n \} \supset [a,b] \Rightarrow \{ x_n \} \]
	- ограниченная последовательность. Тогда по т. Б-В \\
 	$\Rightarrow \exists \{x_n\}$ - подпоследовательность $x_{n_k} \xrightarrow[k\to\infty]{} x_0 \in [a,b]$\\
	т.к. f(x) - непрерывна в $x_0$, то 
	\[ f(x_{n_k}) \xrightarrow[k\to\infty]{} f(x_0) \in \R \]
	\[ \forall k \; |f(x_{n_k})| > n_k \geq k \xrightarrow[k\to\infty]{} \infty \]
	Противоречие
\end{tcolorbox}

\textbf{Теорема 2. Вторая теорема Вейерштрасса} $f(x) \in C[a,b]$, то f(x) достигает на [a,b] max и min значений. 
\begin{tcolorbox}[title=Доказательство от противного]
	По принципу полноты $\R$ по Вейерштрассу
	\[ \exists \sup_{[a,b]} f(x) = M \in \R \]
	По предположению $x \in [a,b]$ $f(x) < M$\\
	т.к. $M = \sup_{[a,b]} f(x), \forall \eps > 0$ 
	\[ \exists x_\eps \in [a,b] \;\;\; f(x_\eps) > M -\eps \]
	\[ \{x_n\}: \forall n, x_n\in[a,b]\ \&\ f(x_n) > m - \frac{1}{n} \]
	Последовательность ограниченна $\Rightarrow_{B-W} \forall k\; \exists \{x_{n_k}\}$ сходящаяся
	\[ x_n \xrightarrow[k\to\infty]{} x_0\in[a,b] \]
	\[ M - \frac{1}{k} \leq M - \frac{1}{n_k} < f(x_{n_k}) < M \Rightarrow f(x_{n_k}) \to M \]
	т.к. f(x) неопределена в $x_0$,
	\[ f(x_{n_k}) \xrightarrow[k\to\infty]{} f(x_0) = M \]
\end{tcolorbox}
\textbf{Теорема 3.} $f(x) \in C[a,b], f(a)\cdot f(b) < 0 \Rightarrow c \in (a,b): f(x) = 0$
\begin{tcolorbox}[title=Доказательство]
	Деление отрезка пополам. 
	\[ [a_0,b_0] \ [a,b] \]
	\begin{enumerate}
		\item 
	\[ c_0 := \frac{a_0+b_0}{2} \]
	Если $f(c_0) \not= 0$, то следующий шаг:\\
	$[a_1,b_1]$ - тот из отрезков $[a_0, c_0]$ и $[c_0, b_0]$ на концах которого у f(x) разные знаки. 
	\[ c_1 := \frac{a_1 + b_1}{2} \]
	... \\
	Возможности: 
	\begin{enumerate}
		\item на некотором шаге $f(c_k) = 0$ 
		\item на каждом шаге $f(c_k) \not=0$ \\
		Тогда \\
		\[ [a_0,b_0] \supset [a_1, b_1] \supset ... \supset [a_n,b_n] \] 
		Вложенные стягивающиеся отрезки
		\[ \cap_n [a_n,b_n] = c\in [a,b] \]
		Проверим, что $f(c) = 0$, если $f(c) \not=0 \Rightarrow \exists U_c$ в которой $f(x)$ сохраняет знак $\exists N$
		\[ \forall n > N \;\; [a_n,b_n] \subset U_c \]
	\end{enumerate}
	\end{enumerate}
\end{tcolorbox}











\end{document}
